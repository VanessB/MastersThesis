\chapter{Теоретические сведения}
\label{chapter:theory} \index{Теория}

В настоящем разделе приведён краткий обзор классических результатов
теории жёстких систем дифференциальных уравнений~\cite{heirer1999solvingode2, lambert1991methods}.
Он необходим для понимания преимуществ неявных численных схем и экспоненциальных интеграторов~---
ключевых составляющих разрабатываемого в рамках данной работы семейства методов для решения жёстких систем.
Также без данного обзора невозможно было бы отделить <<классическую>> жёсткость от <<неклассической>>,
часто наблюдаемой в случае сильно нелинейных уравнений реакций.

Первая секция данной главы посвящена линейной теории устойчивости,
хорошо описывающей поведение численных методов при линейной реакции системы на возмущения численного решения.
На основе линейной теории устойчивости вводится понятие <<классической>> линейной жёсткости.
Вторая секция содержит краткое описание нелинейной теории устойчивости,
обобщающей понятие <<классической>> жёсткости на случай нелинейной реакции системы на возмущения.
В третьей секции приведены результаты, говорящие о необходимости введения нового,
<<неклассического>> определения жёсткости для описания явлений,
не связанных с жёстким характером реакции динамической системы на малые возмущения.


\section{Регулярные функции операторного аргумента}
\label{section:regular_function_operator}

В ходе работы часто будут встречаться выражения,
с которыми проще и удобнее работать в терминах теории регулярных функций операторного аргумента.
В настоящем разделе компактно приведены необходимые результаты из соответствующей области.

\begin{definition}
    \label{definition:regular_function_operator:spectrum}
    Пусть $ \banachspace $~--- банахово пространство,
    $ A \in \linopset(\banachspace) $~--- линейный ограниченный оператор над $ \banachspace $,
    $ \identity $~--- единичный оператор над $ \banachspace $.
    \emph{Спектром оператора $ A $} называется
    $ \sigma(A) = \{ \lambda \mid A - \lambda \identity \; \text{необратим} \} $.
\end{definition}

\begin{definition}[\cite{takesaki2001opalgebras1}, утверждение 2.7]
    \label{definition:regular_function_operator:regular_function_operator}
    Пусть $ A $~--- линейный ограниченный оператор, действующий в банаховом пространстве $ \banachspace $ над $ \complexes $,
    $ \sigma(A) $~--- спектр оператора $ A $, а
    $ f \colon U \to \complexes $~--- регулярная в области $ U \supset \sigma(A) $ функция.
    Тогда можно определить
    \begin{equation}
        \label{eq:regular_function_operator:regular_function_operator}
        f(A) = \frac{1}{2 \pi i} \int\limits_{\Gamma} f(\xi) \left( \xi \identity - A \right)^{-1} d \xi,
    \end{equation}
    где $ \Gamma $~--- произвольный гладкий контур в $ U $ такой,
    что $ \sigma(A) $ целиком находится по левую сторону при положительном обходе $ \Gamma $.
\end{definition}

Для численных методов часто бывает важным понять,
переводит ли некоторая регулярная функция вещественные матрицы в вещественные.
В связи с этим полезны следующие результаты:

\begin{remark}[Производная Виртингера регулярной функции]
    \label{remark:regular_function_operator:Wirtinger_derivative}
    Условие $ \frac{\partial f(z)}{\partial z^*}~=~0 $ эквивалентно условиям Коши-Римана,
    а потому выполнено для всех регулярных функций.
\end{remark}


\begin{lemma}
    \label{lemma:regular_function_operator:complex_conjugate_regularity}
    Если $ f(z) \colon U \to \complexes $ регулярна в симметричной относительно действительной оси области $ U $,
    то $ (f(z^*))^* $ регулярна в той же области.
\end{lemma}

\begin{proof}
    $ \frac{\partial (f(z^*))^*}{\partial z^*} = \left( \frac{\partial f(z^*)}{\partial z} \right)^* =
    \left( \frac{\partial f(z)}{\partial z^*} \right)^* = 0^* = 0 $.
\end{proof}


\begin{corollary}
    \label{corollary:regular_function_operator:sum_with_conjugated_is_constant}
    Пусть $ f $~--- регулярная функция, удовлетворяющая условиям
    леммы~\ref{lemma:regular_function_operator:complex_conjugate_regularity}.
    Тогда $ f(z) - (f(z^*))^* = const $.
\end{corollary}

\begin{proof}
    Согласно лемме~\ref{lemma:regular_function_operator:complex_conjugate_regularity},
    $ g(z) \defeq f(z) - (f(z^*))^* $ регулярна в той же области, что и $ f(z) $.
    При этом
    \[
        \frac{\partial g}{\partial z} = \frac{\partial f}{\partial z} - \frac{\partial (f(z^*))^*}{\partial z} =
        \frac{\partial f}{\partial z} - \left( \frac{\partial f(z^*)}{\partial z^*} \right)^* =
        \frac{\partial f}{\partial z} - \frac{\partial f}{\partial z} = 0
    \]
    Следовательно, $ g(z) = const $.
\end{proof}


\begin{corollary}
    \label{corollary:regular_function_operator:sum_with_conjugated_is_zero}
    Пусть $ f $~--- регулярная функция, удовлетворяющая условиям
    леммы~\ref{lemma:regular_function_operator:complex_conjugate_regularity}
    и $ f(\reals) \subseteq \reals $ ($ f $ переводит вещественные числа в вещественные).
    Тогда $ f(z^*) = (f(z))^* $
\end{corollary}

\begin{proof}
    Достаточно заметить, что $ f(x^*) = (f(x))^* $ для любого действительного числа $ x $
    из области определения $ f(z) $, и воспользоваться
    следствием~\ref{corollary:regular_function_operator:sum_with_conjugated_is_constant}.
\end{proof}


\begin{theorem}
    \label{theorem:regular_function_operator:real_valued_matrix_function}
    Пусть $ M \in \reals^{n \times n} $ и $ f \colon \complexes \to \complexes $
    удовлетворяют определению~\ref{definition:regular_function_operator:regular_function_operator}.
    Пусть также $ f(\reals) \subseteq \reals $.
    Тогда $ f(M) \in \reals^{n \times n} $.
\end{theorem}

\begin{proof}
    Поскольку матрица $ M $ вещественная,
    характеристический многочлен имеет вещественные коэффициенты.
    По теореме о корнях многочлена с вещественными коэффициентами
    $ \sigma(A) $ симметричен относительно вещественной прямой.
    Из условия тогда следует, что можно сузить $ f(z) $
    на симметричную относительно действительной оси область,
    содержащую $ \sigma(A) $.
    Тогда, согласно следствию~\ref{corollary:regular_function_operator:sum_with_conjugated_is_zero}
    получаем $ f(z^*) = (f(z))^* $.

    Рассмотрим симметричный относительно вещественной прямой контур,
    удовлетворяющий~\ref{definition:regular_function_operator:regular_function_operator}.
    Разобъём его на три части: $ \Gamma = \Gamma^+ \sqcup \Gamma^- \sqcup \Gamma^0 $,
    где $ \Gamma^+ \defeq \{ z \in \Gamma \mid \Im z > 0 \} $, $ \Gamma^0 = \Gamma \cap \reals $.
    Тогда выражение~\eqref{eq:regular_function_operator:regular_function_operator} можно переписать в виде
    \begin{align*}
        f(A) &= \frac{1}{2 \pi i} \int\limits_{\Gamma^+} f(\xi) \left( \xi \identity - A \right)^{-1} d \xi +
                \frac{1}{2 \pi i} \int\limits_{\Gamma^-} f(\xi) \left( \xi \identity - A \right)^{-1} d \xi + \\
             &= \frac{1}{2 \pi i} \int\limits_{\Gamma^0} f(\xi) \left( \xi \identity - A \right)^{-1} d \xi,
    \end{align*}
    где последнее слагаемое даёт ноль
    (гладкий симметричный относительно вещественной прямой контур не может иметь пересечение с $ \reals $ ненулевой длины).
    Покажем, что оставшиеся слагаемые комплексно сопряжены:
    %
    \begin{multline*}
        \frac{1}{2 \pi i} \int\limits_{\Gamma^+} f(\xi) \left( \xi \identity - A \right)^{-1} d \xi =
        -\frac{1}{2 \pi i} \int\limits_{\Gamma^-} f(\xi^*) \left( \xi^* \identity - A \right)^{-1} d \xi = \\ =
        -\frac{1}{2 \pi i} \int\limits_{\Gamma^-} \left[ f(\xi) \left( \xi \identity - A \right)^{-1} \right]^{*T} d \xi =
        \left[ \frac{1}{2 \pi i} \int\limits_{\Gamma^-} f(\xi) \left( \xi \identity - A \right)^{-1} d \xi \right]^{*T},
    \end{multline*}
    %
    где было использовано $ A^{*T} \defeq (A^*)^T = A $ и $ f(z^*) = (f(z))^* $.
    Отсюда следует, что итоговая сумма является матрицей с вещественными элементами.
\end{proof}



\section{Линейная теория устойчивости}
\label{section:theory:linear_stability_theory}

При численном решении систем дифференциальных уравнений неизбежны возмущения,
вызванные дискретизацией, а также неточностью машинной арифметики.
Поведение систем и численных методов под действием возмущений во многом определяет
объёмы вычислительных ресурсов, необходимых для получения точного численного решения.
Линейная теория устойчивости численных методов позволяет проанализировать устойчивость
аналитического и численного решения к малым возмущениям в линейном приближении.
Линейный анализ особенно удобен тем,
что позволяет выводить необходимые и достаточные условия устойчивости
из спектральных свойств возникающих линейных операторов.

Рассмотрим задачу Коши
\begin{equation}
    \label{eq:linear_stability_theory:initial_value_problem}
    \begin{dcases}
        \dot{\bvec{x}} = f(t, \bvec{x}) \\
        \bvec{x}(0) = \bvec{x}_0
    \end{dcases}
    \qquad
    \bvec{x} \colon [0; T] \to \banachspace, \qquad f \colon [0;T] \times D \to \banachspace,
\end{equation}
%
где $ D \subseteq \banachspace $,
а $ \banachspace $~--- банахово.
Для простоты анализа и без ограничения общности считаем систему автономной;
это эквивалентно добавлению дополнительной компоненты (времени) с производной,
тождественно равной единице.
%
\begin{equation}
    \label{eq:linear_stability_theory:autonomous_initial_value_problem}
    \begin{dcases}
        \dot{\bvec{x}} = f(\bvec{x}), \\
        \bvec{x}(0) = \bvec{x}_0
    \end{dcases}
\end{equation}
%
Пусть $ f $ дифференцируема в окрестности $ \bvec{x}_0 $.
Рассмотрим соответствующую линеаризованную систему:
%
\begin{equation}
    \label{eq:linear_stability_theory:linearized_initial_value_problem}
    \begin{dcases}
        \dot{\bvec{x}} = f_0 + F_0 \cdot (\bvec{x} - \bvec{x}_0), \\
        \bvec{x}(0) = \bvec{x}_0,
    \end{dcases}
\end{equation}
%
где $ f_0 \defeq f(\bvec{x}_0)$,
а $ \left. \frac{\partial f}{\partial \bvec{x}} \right|_{\bvec{x}_0} = F(\bvec{x}_0) \defeq F_0 $~---
матрица Якоби правой части уравнения в точке $ \bvec{x}_0 $.
Для линеаризованной задачи известно точное решение:
\begin{equation}
    \label{eq:linear_stability_theory:linearized_solution}
    \bvec{x}(t) = \bvec{x}_0 + (\exp(t F_0) - \identity) F_0^{-1} \cdot f_0
\end{equation}

Несмотря на потенциальную вырожденность $ F_0 $,
аппарат функционального анализа позволяет определить~\eqref{eq:linear_stability_theory:linearized_solution}
для всех $ F_0 $.

%\begin{remark}
%    \label{rem:regular_function_diagonalizable_operator}
%    Пусть в условиях опредления \ref{def:regular_function_operator} оператор $ A $ диагонализуемый: $ A = V \Lambda V^{-1} $, $ \Lambda = \diag(\lambda_{\alpha}) $.
%    Тогда $ f(A) = V f(\Lambda) V^{-1} = V \diag \left(f(\lambda_{\alpha}) \right) V^{-1} $.
%\end{remark}

\begin{remark}
    \label{remark:linear_stability_theory:phi_1_function}
    Следующая функция является регулярной во всём $ \complexes $:
    \begin{equation}
        \label{eq:linear_stability_theory:phi_1_function}
        \varphi_1(z) =
        \begin{dcases}
            \frac{e^z - 1}{z}, &\quad z \neq 0 \\
            1, &\quad z = 0
        \end{dcases}
    \end{equation}
\end{remark}

\begin{remark}
    \label{remark:linear_stability_theory:linearized_solution_via_phi_1}
    Решение~\eqref{eq:linear_stability_theory:linearized_solution} представимо в виде
    $ \bvec{x}(t) = \bvec{x}_0 + \varphi_1(t F_0) \cdot t f_0 $.
\end{remark}

Линеаризованная система~\eqref{eq:linear_stability_theory:linearized_initial_value_problem}
и соответствующее аналитическое решение~\eqref{eq:linear_stability_theory:linearized_solution}
могут быть использованы для анализа устойчивости системы к малым возмущениям.
Фактически, линеаризация означает разложение системы на две составляющие:
общий нелинейный тренд, задающий динамику невозмущённого решения,
и линейную часть, отвечающую за реакцию системы на небольшие возмущения.
Предполагая малый вклад нелинейных членов, а также невырожденность $ F_0 $,
можно сделать замену $ \bvec{x}: = \bvec{x} - \bvec{x}_0 + F_0^{-1} \cdot f_0 $
(ноль становится положением равновесия)
и свести задачу к линейному проверочному \emph{уравнению Далквиста} \cite{dahlquist1963special}:
%
\begin{equation}
    \label{eq:linear_stability_theory:Dahlquist_equation}
    \begin{dcases}
        \dfrac{d \bvec{x}}{d t} = F_0 \cdot \bvec{x}, \\
        \bvec{x}(0) = \bvec{x}_0
    \end{dcases}
    \qquad
    \Longleftrightarrow
    \qquad
    \bvec{x}(t) = \exp(t F_0) \cdot \bvec{x}_0
\end{equation}

Для линейных (возможно, многошаговых) численных схем правая часть вида $ F_0 \cdot \bvec{x} $
также позволяет записать переход от текущего к новому шагу численного интегрирования
через действие некоторого линейного оператора:
%
\begin{equation}
    \label{eq:linear_stability_theory:stability_function}
    \bvec{x}^{n+1} = R(\Delta t \cdot F_0) \cdot \bvec{x}^n,
\end{equation}
%
где $ R(z) $~--- \emph{функция устойчивости}, а $ R(\Delta t \cdot F_0) $~--- \emph{матрица перехода}.
Для большинства схем можно рассматривать $ R(z) $ как функцию комплексного переменного,
естественным образом продолженную на матричный аргумент
(см. \ref{definition:regular_function_operator:regular_function_operator}).
Однако есть и исключения:
например, у линейных схем с матричными (не скалярными) весами
функция $ R(z) $ изначально может быть определена только для матричного аргумента.
В классической линейной теории устойчивости такие схемы не рассматриваются.

Приведём несколько примеров линейных численных схем и соответствующих функций устойчивости.
Эти примеры понадобятся при дальнейшем анализе,
а также будут периодически фигурировать в различных сравнениях.
%
{\allowdisplaybreaks
\begin{align}
    \text{\emph{Явный метод Эйлера:}}   && \frac{\bvec{x}^{n+1} - \bvec{x}^n}{\Delta t} &= f(\bvec{x}^n) & R(z) &= 1 + z & \label{eq:forward_Euler} \\
    \text{\emph{Неявный метод Эйлера:}} && \frac{\bvec{x}^{n+1} - \bvec{x}^n}{\Delta t} &= f(\bvec{x}^{n+1}) & R(z) &= \frac{1}{1 - z} & \label{eq:backward_Euler} \\
    \text{\emph{Метод трапеций:}}       && \frac{\bvec{x}^{n+1} - \bvec{x}^n}{\Delta t} &= \frac{1}{2} f(\bvec{x}^n) + \frac{1}{2} f(\bvec{x}^{n+1}) & R(z) &= \frac{2 + z}{2 - z} & \label{eq:trapezoid}
\end{align}
}
%
Общий случай двухшаговой одностадийной схемы:
\begin{equation}
    \label{eq:linear_stability_theory:weighted_two_point}
    \begin{aligned}
        \frac{\bvec{x}^{n+1} - \bvec{x}^n}{\Delta t} &= (1 - \Theta) f(\bvec{x}^n) + \Theta f(\bvec{x}^{n+1}) \\
        R(z) &= (\identity - z \Theta)^{-1}(\identity + z (\identity - \Theta))
    \end{aligned}
\end{equation}
где $ \Theta $, вообще говоря, может быть произвольным обратимым линейным оператором
(в таком случае $ R(\blankarg) $ следует изначально рассматривать как функцию операторного аргумента,
не имеющую прототипа среди функций комплексного переменного.
Если же $ \Theta $~--- число, то данная схема называется \emph{$ \theta $-методом}
(в этом случае используется обозначение $ \Theta = \theta $).


\subsection{Асимптотический анализ}
\label{subsection:linear_stability_theory:asymptotic}

Рассмотрим формально вопросы затухания возмущений аналитического и численного решения с течением времени.
Нас интересуют условия, при которых численное решение сходится к положению равновесия системы дифференциальных уравнений
независимо от начальных возмущений.
Спектральная теория операторов позволяет довольно легко получить необходимые результаты
в терминах асимптотической устойчивости.

\begin{definition}
    \label{definition:asymptotic:spectral_radius_and_abscissa}
    Пусть $ \sigma(A) \subseteq \complexes $~--- спектр линейного оператора $ A $.
    Число $ \displaystyle r(A) = \sup \{|\lambda| \mid \lambda \in \sigma(A) \} $ называется \emph{спектральным радиусом} линейного оператора $ A $,
    а $ \displaystyle s(A) = \sup \{\Re \lambda \mid \lambda \in \sigma(A) \} $~--- \emph{спектральной границей}.
\end{definition}

\begin{theorem}[об отображении спектра; \cite{takesaki2001opalgebras1}, утверждение 2.8]
    \label{theorem:asymptotic:spectral_mapping_theorem}
    Пусть $ A $~--- линейный ограниченный оператор, действующий в банаховом пространстве $ \banachspace $ над $ \complexes $,
    $ f $~--- регулярная в области $ U \supset \sigma(A) $ функция.
    Тогда
    \begin{equation}
        \label{eq:asymptotic:spectral_mapping_theorem}
        \sigma(f(A)) = f(\sigma(A))
    \end{equation}
\end{theorem}

Здесь и далее под нормой оператора будет подразумеваться норма,
подчинённая норме линейного пространства,
в котором действует оператор: $\displaystyle \| A \| = \sup_{\| x \| = 1} \| A x \| $.

\begin{theorem}[формула Бёрлинга-Гельфанда]
    \label{theorem:asymptotic:Beurling-Gelfand_formula}
    Пусть $ A $~--- линейный ограниченный оператор, действующий в банаховом пространстве $ \banachspace $ над $ \complexes $.
    Тогда $ \displaystyle r(A) = \lim_{n \to \infty} \| A^n \|^{\frac{1}{n}} = \inf_{n \in \naturals} \| A^n \|^{\frac{1}{n}} $.
\end{theorem}

\begin{corollary}
    \label{corollary:asymptotic:spectral_radius_norm_bounds}
    Пусть $ A $~--- линейный ограниченный оператор, действующий в банаховом пространстве $ \banachspace $ над $ \complexes $.
    Тогда
    \[
        \forall \varepsilon > 0 \;\; \exists n_0 \in \naturals: \; \forall n > n_0 \quad (r(A))^n \leqslant \| A^n \| < (r(A) + \varepsilon)^n
    \]
\end{corollary}

\begin{proof}
    По теореме \ref{theorem:asymptotic:Beurling-Gelfand_formula}
    \[
        \forall \varepsilon > 0 \;\; \exists n_0 \in \naturals: \; \forall n > n_0 \quad \left| \| A^n \|^{\frac{1}{n}} - r(A) \right| < \varepsilon,
    \]
    причём $ \forall n \in \naturals \;\; (r(A))^n = r(A^n) \leqslant \| A^n \| $.
    Тогда
    \[
        \forall \varepsilon > 0 \;\; \exists n_0 \in \naturals: \; \forall n > n_0 \quad r(A) \leqslant \| A^n \|^{\frac{1}{n}} < r(A) + \varepsilon,
    \]
    откуда получаем искомое неравенство.
\end{proof}

\begin{lemma}
    \label{lemma:asymptotic:operator_exponential_norm_convergence}
    Пусть $ A $~--- линейный ограниченный оператор, действующий в банаховом пространстве $ \banachspace $ над $ \complexes $.
    Тогда
    \[
        s(A) = \inf \left \{ \omega \in \reals: \lim_{t \to +\infty} e^{-\omega t} \cdot \| \exp(t \cdot A) \| = 0 \right \}
    \]
\end{lemma}

\begin{proof}
    Обозначим $ T(t) = \exp(t \cdot A) $.
    Тогда
    \[
        T(t) = \exp((n \Delta t + \tau) \cdot A) = \left( T(\Delta t) \right)^n \cdot T(\tau),
    \]
    где $ \Delta t > 0 $, $ n = \lfloor t / \Delta t \rfloor $, $ \tau = t - n \Delta t \in [0;\Delta t) $.
    В силу ограниченности $ A $ имеем $ \| T(t) \| \leqslant e^{t \cdot \| A \|} $.
    Согласно теореме \ref{theorem:asymptotic:spectral_mapping_theorem}, $ \sigma(T(t)) = \exp(t \cdot \sigma(A)) $,
    из чего следует $ r(T(t)) = e^{t \cdot s(A)} $
    (так как $ \left| e^z \right| = e^{\Re z} $, а экспонента~--- монотонно возрастающая на $ \reals $ функция).
    Применяя следствие \ref{corollary:asymptotic:spectral_radius_norm_bounds} и учитывая, что
    \[
        C^{-1} = e^{-\Delta t \cdot \| A \|} \leqslant \| T(-\tau) \|^{-1} = \| T(\tau)^{-1} \|^{-1} \leqslant \| T(\tau) \| \leqslant e^{\Delta t \cdot \| A \|} = C,
    \]
    получаем, что $ \forall \varepsilon > 0 \;\; \exists n_0 \in \naturals: \; \forall n > n_0 $
    \[
        C^{-1} \cdot e^{n \Delta t \cdot s(A)} \leqslant \| T(t) \| \leqslant C \cdot e^{n \Delta t \cdot (s(A) + \varepsilon)}
    \]
    Наконец, так как $ n \Delta t = t - \tau $, $ \forall \varepsilon > 0 \;\; \exists t_0 > 0: \; \forall t > t_0 $
    \[
        K^{-1} \cdot e^{t \cdot s(A)} \leqslant \| T(t) \| \leqslant K \cdot e^{t \cdot (s(A) + \varepsilon)},
    \]
    где $ K = C \cdot e^{\Delta t \cdot |s(A)|} = e^{\Delta t \cdot (\|A\| + |s(A)|)} > 0 $. %\limto{\Delta t}{0} 1 $.
    Из этого следует, что $ \forall \varepsilon > 0 $
    \[
        \lim_{t \to +\infty} e^{-t \cdot s(A)} \cdot \| T(t) \| \geqslant K^{-1} > 0,
    \]
    \[
        \lim_{t \to +\infty} e^{-t \cdot (s(A) + \varepsilon)} \cdot \| T(t) \| \leqslant \lim_{t \to +\infty} K \cdot e^{-t \cdot \varepsilon / 2} = 0
    \]
    Отсюда по определению точной верхней грани получаем доказываемое утверждение.
\end{proof}

Следствие~\ref{corollary:asymptotic:spectral_radius_norm_bounds}
позволяет по спектру линейного оператора $ A $ оценить асимптотику $ \| A^n \| $,
а лемма~\ref{lemma:asymptotic:operator_exponential_norm_convergence}~--- $ \| \exp(t \cdot A) \| $.

Вернёмся к проверочному уравнению Далквиста \eqref{eq:linear_stability_theory:Dahlquist_equation}.
Нас интересуют равномерные оценки на норму численного решения, получаемого заданным методом при заданном постоянном шаге интегрирования.
Для этого введём следующее определение и утверждение:

\begin{definition}
    \label{definition:asymptotic:stability_region}
    Множество $ \stabreg = \{ z \in \complexes \mid | R(z) | < 1 \} $ называется \emph{областью абсолютной устойчивости} численного метода, обладающего функцией устойчивости $ R(z) $.
    Множество $ \overline{\stabreg} = \{ z \in \complexes \mid | R(z) | \leqslant 1 \} $~--- замыкание области абсолютной устойчивости.%
    \footnote{Здесь неявно предполагается непрерывность $ R(z) $.
    Можно обойтись без этого условия, однако тогда некорректно называть $ \overline{\stabreg} $ замыканием $ \stabreg $.}
\end{definition}

\begin{statement}
    \label{statement:asymptotic:linear_numerical_stability}
    Пусть численное решение уравнения \eqref{eq:linear_stability_theory:Dahlquist_equation}
    ищется интегрированием с постоянным шагом $ \Delta t $
    при помощи метода, обладающего функцией устойчивости $ R(z) $ и соответствующей областью абсолютной устойчивости $ \stabreg $.
    Пусть также $ R(z) $ регулярна в области $ U \supset \Delta t \cdot \sigma(F_0) $.
    Тогда $ \bvec{x}^n = \left( R(\Delta t \cdot F_0) \right)^n \cdot \bvec{x}_0 $, и
    \begin{align}
        \Delta t \cdot \sigma(F_0) \subseteq \stabreg \qquad \Longleftrightarrow \qquad & \| \left( R(\Delta t \cdot F_0) \right)^n \| \limto{n}{\infty} 0 \\
        \Delta t \cdot \sigma(F_0) \subseteq \complexes \setminus \overline{\stabreg} \qquad \Longrightarrow \qquad & \| \left( R(\Delta t \cdot F_0) \right)^n \| \limto{n}{\infty} \infty
    \end{align}
\end{statement}

\begin{proof}
    В силу \eqref{eq:linear_stability_theory:stability_function} имеем первое утверждение:
    $ \bvec{x}^n = \left( R(\Delta t \cdot F_0) \right)^n \cdot \bvec{x}_0 $.
    Далее заметим, что по теореме \ref{theorem:asymptotic:spectral_mapping_theorem}
    \[
        \sigma\left( R(\Delta t \cdot F_0) \right) = R\left( \Delta t \cdot \sigma(F_0) \right)
    \]
    Отсюда следует, что
    \begin{align}
        \Delta t \cdot \sigma(F_0) \subseteq \stabreg \qquad \Longleftrightarrow \qquad & r\left( R(\Delta t \cdot F_0) \right) < 1 \\
        \Delta t \cdot \sigma(F_0) \subseteq \complexes \setminus \overline{\stabreg} \qquad \Longleftrightarrow \qquad & r\left( R(\Delta t \cdot F_0) \right) > 1
    \end{align}
    Наконец, применяя следствие \ref{corollary:asymptotic:spectral_radius_norm_bounds}, завершаем доказательство утверждения.
\end{proof}

В теории линейной устойчивости важная роль отводится \emph{A-ус\-той\-чи\-вос\-ти}~---
свойству численного решения проверочного уравнения Далквиста асимптотически не возрастать по норме,
если асимптотически не возрастает норма истинного решения.
Если к тому же при увеличении шага интегрирования норма численного решения на следующей итерации (или, быть может, через некоторе заранее известное число итераций)
также стремится к нулю, то говорят об \emph{L"=устойчивости}.
Дадим формальное определение.

\begin{definition}
    \label{definition:asymptotic:A_stability}
    Численный метод называется \emph{A"=устойчивым} в случае,
    если $ \complexes^- \defeq \{ z \in \complexes \mid \Re z < 0 \} \subseteq \stabreg $.
\end{definition}

\begin{definition}
    \label{definition:asymptotic:L_stability}
    Численный метод называется \emph{L"=устойчивым} в случае,
    если он A"=устойчив и выполнено $ \displaystyle\lim_{\Re z \to -\infty} R(z) = 0 $.
\end{definition}

\begin{statement}
    \label{statement:asymptotic:A_stability_criterion}
    Пусть $ R(z) $ регулярна в $ \complexes^- $.
    Соответствующий численный метод A"=устойчив тогда и только тогда, когда $ \forall F_0 $ выполнено
    \[
        \| \exp(t \cdot F_0) \| \limto{t}{+\infty} 0 \qquad \Longrightarrow \qquad \forall \Delta t > 0 \quad \left \| \left( R(\Delta t \cdot F_0) \right)^n \right\| \limto{n}{\infty} 0
    \]
\end{statement}

\begin{proof}
    В силу леммы \ref{lemma:asymptotic:operator_exponential_norm_convergence} импликацию из утверждения можно переписать в виде
    \[
        \sigma(F_0) \subseteq \complexes^- \quad \Longrightarrow \quad \forall \Delta t > 0 \quad \left \| \left( R(\Delta t \cdot F_0) \right)^n \right\| \limto{n}{\infty} 0
    \]
    Если также воспользоваться \ref{statement:asymptotic:linear_numerical_stability}, получаем
    \[
        \sigma(F_0) \subseteq \complexes^- \quad \Longrightarrow \quad \forall \Delta t > 0 \quad \Delta t \cdot \sigma(F_0) \subseteq \stabreg,
    \]
    что при произвольном $ F_0 $ равносильно $ \complexes^- \subseteq \stabreg $.
    Это даёт определение \ref{definition:asymptotic:A_stability}.
\end{proof}

\begin{statement}
    \label{statement:asymptotic:L_stability_property}
    Пусть $ R(z) $ регулярна в $ \complexes^- $.
    Если соответствующий численный метод L"=устойчив, то $ \forall F_0 $ выполнено
    \[
        \| \exp(t \cdot F_0) \| \limto{t}{+\infty} 0 \qquad \Longrightarrow \qquad r \left( R(\Delta t \cdot F_0) \right) \limto{\Delta t}{+\infty} 0
    \]
\end{statement}

\begin{proof}
    Аналогично доказательству утверждения \ref{statement:asymptotic:A_stability_criterion} перепишем импликацию в виде
    \[
        \sigma(F_0) \subseteq \complexes^- \quad \Longrightarrow \quad r \left( R(\Delta t \cdot F_0) \right) \limto{\Delta t}{+\infty} 0
    \]
    В силу теоремы \ref{theorem:asymptotic:spectral_mapping_theorem} это эквивалентно
    \[
        \sigma(F_0) \subseteq \complexes^- \quad \Longrightarrow \quad \sup \{ |\lambda| \mid \lambda \in R(\Delta t \cdot \sigma(F_0)) \} \limto{\Delta t}{+\infty} 0,
    \]
    что при произвольном $ F_0 $ равносильно $ \forall z \in \complexes^- \; | R(\Delta t \cdot z) | \limto{\Delta t}{+\infty} 0 $.
    Это верно для L"=устойчивых методов по определению \ref{definition:asymptotic:L_stability}.
\end{proof}

Стоит отметить, что утверждение \ref{statement:asymptotic:L_stability_property},
в отличие от \ref{statement:asymptotic:A_stability_criterion},
сформулировано в форме признака, а не критерия.
Также в нём получено лишь утверждение о пределе спектрального радиуса, а не нормы.
На деле это означает, что получаемая для L"=устойчивого метода матрица перехода
с увеличением размера шага становится <<почти нильпотентной>>.
Если $ F_0 $ диагонализуемая, то из стремления к нулю спектрального радиуса матрицы перехода
будет автоматически следовать и стремление к нулю её нормы.

На рисунках~\ref{fig:asymptotic:linear_instability_example},
\ref{fig:asymptotic:linear_instability_example_2}
проиллюстрировано поведение явного метода Эйлера (не A"=устойчивый),
метода трапеций (A"=устойчивый, но не L"=устойчивый)
и неявного метода Эйлера (L"=устойчивый) при разных значениях $ \Delta t \cdot F_0 $ в одномерной задаче Далквиста
(на втором графике явный метод Эйлера расходится).

\begin{figure}[ht!]
    \centering
    \small
    \begin{gnuplot}[terminal=tikz, terminaloptions={color size 16cm,6cm fontscale 0.9}]
        load './gnuplot/common.gp'

        set style increment default
        set style data lines
        set xlabel  '$ t $'
        set xrange  [ 0 : 10 ] noreverse writeback
        set ylabel  '$ x(t) $' #rotate by 0
        set yrange  [ * : * ] noreverse writeback

        set key width -16

        # Параметры.
        z = -1.5
        N = 5                    # Число точек.
        T = 9.0                  # Время интегрирования.
        lamb = z * (N - 1) / T   # Показатель экспоненты.

        load './gnuplot/Dahlquist.gp'

        set xtics 1
        set xzeroaxis lw 2

        plot Trapezoid using (times[$1]):(Trapezoid[$1]) with linespoints t 'метод трапеций' ps 2, \
             BackwardEuler using (times[$1]):(BackwardEuler[$1]) with linespoints t 'неявный метод Эйлера' ps 2, \
             ForwardEuler using (times[$1]):(ForwardEuler[$1]) with linespoints t 'явный метод Эйлера' ps 2, \
             f(x) t 'точное решение' ls 200
    \end{gnuplot}
    \caption{Поведение простейших численных методов при решении одномерного уравнения Далквиста ($ \Delta t \cdot F_0 = -1.5 $).}
    \label{fig:asymptotic:linear_instability_example}
\end{figure}

\begin{figure}[ht!]
    \centering
    \small
    \begin{gnuplot}[terminal=tikz, terminaloptions={color size 16cm,6cm fontscale 0.9}]
        load './gnuplot/common.gp'

        set style increment default
        set style data lines
        set xlabel  '$ t $'
        set xrange  [ 0 : 10 ] noreverse writeback
        set ylabel  '$ x(t) $' #rotate by 0
        set yrange  [ * : * ] noreverse writeback

        set key width -16

        # Параметры.
        z = -15.0
        N = 5                    # Число точек.
        T = 9.0                  # Время интегрирования.
        lamb = z * (N - 1) / T   # Показатель экспоненты.

        load './gnuplot/Dahlquist.gp'

        set xtics 1
        set xzeroaxis lw 2

        plot Trapezoid using (times[$1]):(Trapezoid[$1]) with linespoints t 'метод трапеций' ps 2, \
             BackwardEuler using (times[$1]):(BackwardEuler[$1]) with linespoints t 'неявный метод Эйлера' ps 2, \
             f(x) t 'точное решение' ls 200
    \end{gnuplot}
    \caption{Поведение простейших численных методов при решении одномерного уравнения Далквиста ($ \Delta t \cdot F_0 = -15 $).}
    \label{fig:asymptotic:linear_instability_example_2}
\end{figure}



\subsection{Неасимптотический анализ}
\label{subsection:linear_stability_theory:non_asymptotic}

В предыдущем разделе были приведены основные результаты асимптотической линейной теории устойчивости.
Все они в значительной степени опираются на спектральный анализ.
С одной стороны, это облегчает исследование свойств численных методов,
так как теорема об отображении спектра позволяет определить
асимптотические свойства матрицы перехода по функции устойчивости метода.
С другой стороны, спектральный анализ не позволяет получить оценки,
справедливые в течение всего рассматриваемого времени,
включая начальный этап эволюции линейной системы.

Для вывода более строгих в указанном смысле оценок вводится \emph{логарифмическая норма}:

\begin{definition}
    \label{definition:non_asymptotic:logarithmic_norm}
    Пусть $ A $~--- линейный ограниченный оператор, действующий в банаховом пространстве $ \banachspace $ над $ \complexes $.
    Число
    \begin{equation}
        \label{eq:logarithmic_norm}
        \lognorm{A} = \lim_{h \to +0} \frac{\| \identity + h \cdot A \| - 1}{h}
    \end{equation}
    называется \emph{логарифмической нормой} оператора $ A $.
\end{definition}

Далее приведём без доказательства несколько утверждений,
связывающих логарифмическую норму с теорией линейной устойчивости.
Все утверждения взяты из работ \cite{soderlind2006lognorm, lambert1991methods}.

\begin{statement}
    \label{statement:non_asymptotic:logarithmic_norm_properties}
    Пусть $ A $, $ B $~--- линейные операторы, действующие в конечномерном банаховом пространстве $ \banachspace $ над $ \complexes $.
    Тогда верно следующее:
    \begin{enumerate}[itemsep=0em]
        \item $ \lognorm{A} $ определена;
        \item $ \lognorm{A} \leqslant \| A \| $;
        \item $ \lognorm{\gamma \cdot A} = \gamma \cdot \lognorm{A} $ для любого $ \gamma > 0 $;
        \item $ \lognorm{A + z \cdot \identity} = \lognorm{A} + \Re z $;
        \item $ \lognorm{A + B} \leqslant \lognorm{A} + \lognorm{B} $;
        \item $ s(A) \leqslant \lognorm{A} $; \label{item:non_asymptotic:logarithmic_norm_and_spectral_bound}
        \item $ \| \exp(t \cdot A) \| \leqslant e^{t \cdot \lognorm{A}} $ для любого $ t \geqslant 0 $. \label{item:non_asymptotic:logarithmic_norm_exponential_bound}
    \end{enumerate}
\end{statement}

Пункт \ref{item:non_asymptotic:logarithmic_norm_and_spectral_bound}
связывает логарифмическую норму со спектральными свойствами $ A $.
Пункт \ref{item:non_asymptotic:logarithmic_norm_exponential_bound}
даёт справедливую в течение всего неотрицательного времени оценку на оператор эволюции,
соответствующий оператору $ A $;
однако в силу пункта \ref{item:non_asymptotic:logarithmic_norm_and_spectral_bound},
а также леммы \ref{lemma:asymptotic:operator_exponential_norm_convergence},
данная оценка может не являться асимптотически оптимальной.

Для иллюстрации различий между $ s(A) $, $ \lognorm{A} $ и $ \| \exp(t \cdot A) \| $
можно рассмотреть пространство $ \reals^2 $ с $ l_1 $-нормой и оператор
\begin{equation}
    \label{eq:non_asymptotic:matrix_1}
    A =
    \begin{pmatrix}
        -1 & \alpha \\
        0 & -2
    \end{pmatrix},
    \qquad
    \exp(t \cdot A) =
    \begin{pmatrix}
        e^{-t} & \alpha e^{-t} (1 - e^{-t}) \\
        0 & e^{-2 t}
    \end{pmatrix}
\end{equation}
Нетрудно получить
\[
    s(A) = -1, \quad
    \lognorm{A} =
    \begin{cases}
        -1, &\; |\alpha| < 1 \\
        |a| - 2, &\; |\alpha| \geqslant 1
    \end{cases},
\]
\[
    \| \exp(t \cdot A) \|_1 =
    \begin{cases}
        e^{-1}, &\; |\alpha| < 1 \\
        |\alpha| e^{-t}(1 - e^{-t}) + e^{-2t}, &\; |\alpha| \geqslant 1
    \end{cases}
\]
Видно, что при $ |\alpha| > 1 $ оценка $ |\alpha| \exp(t \cdot s(A)) $ асимптотически оптимальна,
но в окрестности нуля отличается от $ \| \exp(t \cdot A) \| $ в $ |\alpha| $ раз;
с другой стороны, оценка $ \exp(t \cdot \lognorm{A}) $ точна в нуле,
но отличается от $ \| \exp(t \cdot A) \| $ асимптотически.
Разница становится еще более существенной в случае Жордановой клетки:
\begin{equation}
    \label{eq:non_asymptotic:matrix_2}
    A =
    \begin{pmatrix}
        -1 & \alpha \\
        0 & -1
    \end{pmatrix},
    \qquad
    \exp(t \cdot A) =
    \begin{pmatrix}
        e^{-t} & \alpha \, t e^{-t} \\
        0 & e^{-t}
    \end{pmatrix}
\end{equation}
\[
    s(A) = -1, \quad
    \lognorm{A} = |a| - 1, \quad
    \| \exp(t \cdot A) \|_1 = e^{-t} (1 + |\alpha| t)
\]
Здесь при $ \alpha \neq 0 $ выражение $ C \cdot \exp(t \cdot s(A)) $ не является оценкой сверху на $ \| \exp(t \cdot A) \| $
для любого $ t \geqslant 0 $ ни при каких $ C \in \reals $.
Также в обоих случаях $ \exp(t \cdot \lognorm{A}) $ может монотонно возрастать при определённых значениях $ \alpha $,
хотя норма оператора эволюции всегда стремится к нулю при $ t \to +\infty $.
Графики оценок для обоих примеров при $ \alpha = 3/2 $ приведены на рис~\ref{fig:non_asymptotic:matrix_norm_examples}.

\begin{figure}[ht!]
    \centering
    \small
    \begin{subfigure}[t]{0.5\textwidth}
        \centering
        \captionsetup{aboveskip=-\baselineskip}
        \begin{gnuplot}[terminal=tikz, terminaloptions={color size 8.0cm,5.0cm fontscale 0.9}]
            load "./gnuplot/common.gp"

            set yrange [0:1.6]
            set xlabel "$ t $"

            alpha = 1.5
            spectral_bound = -1
            logarithmic_norm = (abs(alpha) < 1.0) ? -1 : abs(alpha) - 2.0

            eonorm(x) = (abs(alpha) < 1.0 ? exp(-x) : abs(alpha) * exp(-x) * (1.0 - exp(-x)) + exp(-2.0 * x))
            spectral_ub(x) = abs(alpha) * exp(spectral_bound * x)
            logarithmic_norm_ub(x) = exp(logarithmic_norm * x)

            plot [0:3] logarithmic_norm_ub(x) t "$ \\exp(t \\cdot \\lognorm{A}) $", \
                       spectral_ub(x) t "$ |\\alpha| \\exp(t \\cdot s(A)) $", \
                       eonorm(x) ls 200 t "$ \\| \\exp(t \\cdot A) \\|_1 $"
        \end{gnuplot}
        \caption{Оператор из примера~\eqref{eq:non_asymptotic:matrix_1}.}
    \end{subfigure}%
    \begin{subfigure}[t]{0.5\textwidth}
        \centering
        \captionsetup{aboveskip=-\baselineskip}
        \begin{gnuplot}[terminal=tikz, terminaloptions={color size 8.0cm,5.0cm fontscale 0.9}]
            load "./gnuplot/common.gp"

            set yrange [0:2.0]
            set xlabel "$ t $"

            alpha = 1.5
            spectral_bound = -1
            logarithmic_norm = abs(alpha) - 1.0

            eonorm(x) = exp(-x) * (1 + abs(alpha) * x)
            spectral_ub(x) = 3.0 * abs(alpha) * exp(spectral_bound * x)
            logarithmic_norm_ub(x) = exp(logarithmic_norm * x)

            plot [0:3] logarithmic_norm_ub(x) t "$ \\exp(t \\cdot \\lognorm{A}) $", \
                       spectral_ub(x) t "$ 3 |\\alpha| \\exp(t \\cdot s(A)) $", \
                       eonorm(x) ls 200 t "$ \\| \\exp(t \\cdot A) \\|_1 $"
        \end{gnuplot}
        \caption{Оператор из примера~\eqref{eq:non_asymptotic:matrix_2}.}
    \end{subfigure}
    \caption{Графики $ l_1 $-норм операторов эволюции
    из уравнений~\eqref{eq:non_asymptotic:matrix_1} и \eqref{eq:non_asymptotic:matrix_2},
    а также соответствующих оценок;
    $ \alpha = 3/2 $.}
    \label{fig:non_asymptotic:matrix_norm_examples}
\end{figure}

\FloatBarrier


\section{Нелинейная теория устойчивости}
\label{section:theory:nonlinear_stability_theory}

Линейная теория устойчивости является удобным инструментом для получения достаточных признаков устойчивости численного решения.
Однако используемое при анализе линейное приближение правой части системы
имеет вполне определённые границы применимости:
\begin{enumerate}
    \item
        Линейный анализ устойчивости некорректно применять при наличии возмущений,
        величина или характер которых ставит под сомнение точность линеаризации системы.
    \item
        Линейный анализ устойчивости некорректно применять,
        если характерное время линейной реакции на малые возмущения не является пренебрежимо малым в сравнении с характерными масштабами времени,
        на которых меняется матрица Якоби правой части системы.
\end{enumerate}
В главе 7 работы \cite{lambert1991methods} даны примеры систем,
применение линейного анализа к которым даёт некорректные результаты
(по крайней мере, если ограничиваться лишь спектральными признаками и не использовать оценки,
приведённые в разделе~\ref{subsection:linear_stability_theory:non_asymptotic}).

Результаты линейной теории устойчивости, однако, можно обобщить на произвольные системы,
в том числе и на не поддающиеся линеаризации.
Для этого следует вспомнить, что уравнение Далквиста~\eqref{eq:linear_stability_theory:Dahlquist_equation}
используется для описания реакции системы на малые возмущения.
Убывание решений уравнения Далквиста соответствует сближению
возмущённого и невозмущенного решения исходного уравнения с течением времени~--- \emph{сжиманию}.
Сжимание можно обобщить на решения произвольной системы,
а не только линеаризованной:

\begin{definition}
    \label{definition:nonlinear_stability_theory:contractivity}
    Пусть $ \bvec{x}(t) $ и $ \bvec{y}(t) $~--- решения системы $ \dot{\bvec{x}} = f(t, \bvec{x}) $.
    Данные решения называются \emph{сжимающимися} на отрезке $ [a; b] $ в случае
    \[
        \forall t_1, t_2: \; a \leqslant t_1 \leqslant t_2 \leqslant b \qquad \| \bvec{x}(t_2) - \bvec{y}(t_2) \|
        \leqslant \| \bvec{x}(t_1) - \bvec{y}(t_1) \|
    \]
\end{definition}

Аналогичное определение можно ввести и для численных решений (и, вообще говоря, для любых последовательностей):

\begin{definition}
    \label{definition:nonlinear_stability_theory:contractivity_sequence}
    Пусть $ \{\bvec{x}_n\}_{n \in \naturals_0} $ и $ \{\bvec{y}_n\}_{n \in \naturals_0} $~---
    последовательности элементов некоторого линейного нормированного пространства.
    Данные последовательности называются \emph{сжимающимися} на отрезке $ [a; b] $ в случае
    \[
        \forall n_1, n_2: \; a \leqslant n_1 \leqslant n_2 \leqslant b \quad \| \bvec{x}_{n_2} - \bvec{y}_{n_2} \| \leqslant \| \bvec{x}_{n_1} - \bvec{y}_{n_1} \|
    \]
\end{definition}

Сжимание решений говорит об устойчивости системы к возмущениям.
Сжимание численных решений~--- об устойчивости численного метода.
Для систем вида $ \dot{\bvec{x}} = f(t, \bvec{x}) $ известно достаточное условие
сжимаемости точных решений~\cite{auzinger1990note, auzinger1993modern}.

\begin{definition}
    \label{definition:nonlinear_stability_theory:one-sided_lipschitz}
    Пусть $ \hilbertspace $~--- гильбертово пространство над $ \complexes $.
    Функция $ f(t, \bvec{x}) = f: \reals \times D \to \hilbertspace $
    (где $ D \subseteq \hilbertspace $~--- выпуклая область)
    и система $ \dot{\bvec{x}} = f(t, \bvec{x}) $ называются \emph{односторонне липшициевыми} на отрезке $ [a; b] $ в случае
    \[
        \exists \nu(t): \; \forall \bvec{x}, \bvec{y} \in D, \; \forall t \in [a, b] \quad
        \Re \dotprod{f(t, \bvec{x}) - f(t, \bvec{y})}{\bvec{x} - \bvec{y}} \leqslant \nu(t) \| \bvec{x} - \bvec{y} \|^2
    \]
\end{definition}

\begin{definition}
    \label{definition:nonlinear_stability_theory:dissipative}
    Пусть в определении \ref{definition:nonlinear_stability_theory:one-sided_lipschitz} $ \nu(t) \leqslant 0 $.
    Тогда функция $ f(t, \bvec{x}) $ и система $ \dot{\bvec{x}} = f(t, \bvec{x}) $ называются \emph{диссипативными}.
\end{definition}

\begin{statement}[достаточный признак сжимаемости]
    \label{statement:nonlinear_stability_theory:contractivity_condition}
    Пусть система $ \dot{\bvec{x}} = f(t, \bvec{x}) $ диссипативна на $ [a; b] $.
    Тогда все её решения являются сжимающимися на $ [a; b] $.
\end{statement}

\begin{remark}
    \label{remark:nonlinear_stability_theory:linear_one-sided_lipschitz}
    Линейная функция $ f(t, \bvec{x}) = A(t) \cdot \bvec{x} + f_0(t) $ односторонне липшициева на всей области определения $ A(t) $
    с наименьшим возможным коэффициентом $ \nu(t) = \lognorm{A(t)} $.
\end{remark}

Замечание \ref{remark:nonlinear_stability_theory:linear_one-sided_lipschitz}
позволяет связать линейную теорию устойчивости с нелинейной
(см. раздел~\ref{subsection:linear_stability_theory:non_asymptotic}).
Его естественным продолжением является следующая теорема:

\begin{theorem}[Далквист, 1959]
    \label{thm:nonlinear_to_linear}
    Пусть $ \hilbertspace $~--- гильбертово пространство над $ \complexes $.
    Пусть $ f(t, \bvec{x}) = f: \reals \times D \to \hilbertspace $,
    где $ D \subseteq \hilbertspace $~--- выпуклая область.
    Пусть также существуют $ a, b \in \reals $ и $ \nu(t) $~--- такая кусочно-непрерывная функция, что
    \[
        \forall t \in [a; b], \; \forall \bvec{x} \in D \quad \lognorm{\frac{\partial f}{\partial \bvec{x}}(t, \bvec{x})} \leqslant \nu(t)
    \]
    Тогда, если $ \bvec{x}(t) $ и $ \bvec{y}(t) $~--- два решения системы $ \dot{\bvec{x}} = f(t, \bvec{x}) $,
    то для любых $ t_1, t_2 \in \reals: \; a \leqslant t_1 \leqslant t_2 \leqslant b $ выполнено
    \[
        \| \bvec{x}(t_2) - \bvec{y}(t_2) \| \leqslant \exp\left( \int\limits_{t_1}^{t_2} \nu(s) ds \right) \| \bvec{x}(t_1) - \bvec{y}(t_1) \|
    \]
\end{theorem}

Наконец, приведём достаточное условие сжимаемости численных решений,
позволяющее распространить анализ устойчивости численных методов на нелинейные системы.

\begin{theorem}[Далквист, 1978]
    Пусть рассматривается конечномерное гильбертово пространство над $ \reals $.
    Если численный метод A"=устойчивый, то существует норма (называемая \emph{G-нормой}),
    в которой любые два численных решения диссипативной на $ [a; b] $ системы будут сжимающимися на $ [a; b] $.
\end{theorem}

Сразу отметим, что известны и другие признаки сжимаемости численных решений,
среди которых есть и не требующие введение другой нормы.
Однако для общего анализа их включение в данную работу не является необходимым.
%В силу специфики численных методов, которые будут предложены в данной работе,
%упомянутые признаки не являются релевантными.

Приведённые теоремы позволяет обосновать линейный анализ устойчивости даже для нелинейных систем.
Стоит, однако, отметить, что это возможно исключительно благодаря ослаблению утверждений до формы достаточных условий,
а также благодаря использованию логарифмической нормы вместо спектральной границы.
Напомним, что первая даёт справедливую в течение всего времени оценку на норму оператора эволюции линейной системы,
а вторая~--- только асимптотическую, пусть и, зачастую, более строгую.



\section{Жесткие системы дифференциальных уравнений}
\label{section:stiffness}

На практике часто встречаются задачи,
численное решение которых возможно только с малым шагом интегрирования,
либо с использованием сложных и дорогих численных методов.
Такие задачи традиционно называются \emph{жёсткими}.
Как указано в~\cite{heirer1999solvingode2, lambert1991methods}, существует несколько определений жёсткости,
каждое из которых обладает своими достоинствами и недостатками.
Приведём здесь одно из них:

\begin{definition}
    \label{definition:stiffness:stiffness}
    Система вида $ \dot{\bvec{x}} = f(t, \bvec{x}) $ называется \emph{жёсткой} в том случае,
    если для получения корректного (то есть в заданной степени близкого к точному решению)
    численного решения необходимо использовать шаг интегрирования,
    много меньший характерных масштабов времени, на которых меняется точное решение.
\end{definition}

Данное определение слишком общее и не отвечает на вопросы о природе ограничений на шаг интегрирования.
На основе всего вышеизложенного анализа мы дадим более узкое,
но в некоторой степени и более информативное определение жёсткости.


\subsection{Классическая жёсткость}
\label{subsection:stiffness:classical_stiffness}

Как видно из приведённого обзора классических результатов теории устойчивости,
спектр матрицы Якоби правой части $ F \defeq \frac{\partial f}{\partial x} $
может задавать определённые ограничения на шаг интегрирования.
Действительно, если $ \lognorm{F} < 0 $, но численный метод не A"=устойчив,
полученное с его помощью решение может вести себя некорректно при некоторых $ F $ и $ \Delta t $.
В частности, если $ \Delta t \cdot \sigma(F) \not\subseteq \stabreg $,
то небольшие возмущения численного решения могут неограниченно усиляться,
в то время как возмущения точного решения будут, наоборот, затухать.
Зачастую область устойчивости не A"=устойчивых методов ограничена
(в частности, это верно для всех явных линейных численных методов)
или содержит лишь некоторый подсектор $ \complexes^- $,
поэтому ограничение на шаг интегрирования оказывается ограничением сверху.
Таким образом, спектральные свойства $ F $ обуславливают максимально допустимый шаг численного интегрирования.

Использование A"=устойчивых методов позволяет избежать указанных проблем.
Однако A"=устойчивость не гарантирует, что возмущения численного решения будут затухать с той же скоростью,
что и возмущения точного;
возможен даже случай $ \displaystyle \lim_{\Re z \to -\infty} |R(z)| = 1 $,
соответствующий слабо затухающим осцилляциям численного решения там,
где точное решение обладает сильным эффектом демпфирования
(см. рисунок~\ref{fig:asymptotic:linear_instability_example_2}).
Если требуется адекватно воспроизводить сильное затухание за конечное число шагов при увеличении $ \Delta t $,
следует пользоваться L"=устойчивыми методами.

Поскольку любой линейный A"=устойчивый метод является неявным~\cite{lambert1991methods},
платой за повышенную устойчивость численного решения
является необходимость на каждом шаге интегрирования решать (в общем случае нелинейное) уравнение на корни невязки.
Ситуацию осложняет следующая теорема,
требующая в случае одностадийных схем делать выбор между устойчивостью, высоким порядком аппроксимации и линейностью схемы:

\begin{theorem}[второй барьер Далквиста]
    \label{theorem:classical_stiffness:Dahlquist_second_barrier}
    Не существует A"=устойчивых линейных многошаговых одностадийных схем с порядком аппроксимации выше второго.
\end{theorem}

Многостадийные неявные методы требуют решения уже нескольких алгебраических уравнений
на каждом шаге интегрирования,
что еще более увеличивает их вычислительную сложность.

Приведённые выше рассуждения показывают,
что определённые системы дифференциальных уравнений могут обладать свойствами,
вынуждающими либо использовать малый шаг интегрирования,
либо задействовать дорогостоящие неявные численные методы,
устойчивые в смысле определений~\ref{definition:asymptotic:A_stability} и \ref{definition:asymptotic:L_stability}.
%Такие системы попадают под общее определение жёсткости~\ref{definition:stiffness:stiffness}.
%Конкретизируем его, используя классические результаты теории устойчивости.
Это позволяет конкретизировать определение жёсткости~\ref{definition:stiffness:stiffness},
используя классические результаты теории устойчивости.

\begin{definition}
    \label{definition:classical_stiffness:linear_and_nonlinear_timescales}
    Обозначим характерное время изменения $ F = \frac{\partial f}{\partial \bvec{x}}(\bvec{x}(t)) $ как~$ \taunonlin $,
    а $ 1 / r(F) $~---
    характерное время реакции линеаризованной системы на небольшие возмущения~--- как~$ \taulin $.
\end{definition}

Рассмотрим случай $ \taulin \ll \taunonlin $,
позволяющий применять линейный анализ устойчивости.

\begin{definition}
    \label{definition:classical_stiffness:linear_stiffness}
    Система вида $ \dot{\bvec{x}} = f(\bvec{x}) $ называется \emph{линейно жёсткой} в том случае,
    если характерное время реакции линеаризованной системы на небольшие возмущения $ \taulin $
    много меньше характерных масштабов времени,
    на которых ищется решение.
\end{definition}

Зачастую численное решение ищется на интервалах,
сопоставимых с характерными временн\'{ы}ми масштабами наиболее медленно протекающих процессов, описываемых системой
(это действительно так, если в численном решении требуется полноценно отобразить всю динамику системы).
Также для численных методов с ограниченной областью устойчивости условие $ \Delta t \cdot \sigma(F) \subseteq \stabreg $ влечет $ \Delta t \sim \taulin $.
Тогда определение~\ref{definition:classical_stiffness:linear_stiffness} оказывается
частным случаем определения~\ref{definition:stiffness:stiffness},
причём необходимость выбора малого шага оказывается обусловленной <<жёстким>> линейным поведением системы в окрестности истинных решений.

Стоит также отметить, что в случае линейных или слабо нелинейных
(то есть линеаризацию которых можно долго считать достаточно точной) систем
характерные масштабы времени наиболее медленно протекающих процессов задаются наименьшими по модулю элементами спектра матрицы $ F $.
Тогда для линейно жёстких систем спектр $ F $ распадается на две части: \emph{ведущую} и \emph{паразитическую}.

\begin{definition}
    \label{definition:classical_stiffness:driving_and_parasitic_spectrum}
    Будем называть спектр оператора $ A $ \emph{распадающимся на ведущую и паразитическую части}, если
    существует расщепление $ \sigma(A) = \sigma_d(A) \sqcup \sigma_p(A) $ такое, что
    \[
        r_d(A) \defeq \sup \{ | \lambda | \mid \lambda \in \sigma_d(A) \} \ll
        \inf \{ | \lambda | \mid \lambda \in \sigma_p(A) \} \defeq b_p(A)
    \]
\end{definition}

Более того, в случае седловых задач доля паразитического спектра может иметь положительную действительную часть.
Это может вызывать <<взрывное>> поведение даже A"=устойчивых методов,
ведь в таком случае попадание $ \Delta t \cdot \sigma(F) $ в область устойчивости не гарантируется.
В задачах, где известно, что такое поведение точного решение нефизично и отсутствует в виду выбора правильных начальных условий,
данное свойство численного решения является нежелательным.

Нелинейный анализ устойчивости позволяет обобщить определение~\ref{definition:classical_stiffness:linear_stiffness}
на случай, когда $ \taunonlin \not\ll \taulin $.
Для этого требуется рассмотреть характерное время затухания возмущений в диссипативных системах
и сравнить его с характерным временем эволюции невозмущённого решения.

\begin{definition}
    \label{definition:classical_stiffness:dissipation_timescale}
    Обозначим $ 1 / \nu(t) $~--- характерное время релаксации возмущений в диссипативной системе
    $ \dot{\bvec{x}} = f(t, \bvec{x}) $~--- как~$ \taudiss $.
\end{definition}

\begin{definition}
    \label{definition:classical_stiffness:classical_stiffness}
    Диссипативная система вида $ \dot{\bvec{x}} = f(\bvec{x}) $ называется \emph{(классически) жёсткой} в том случае,
    если характерное время реакции системы на небольшие возмущения $ \taudiss $
    много меньше характерных масштабов времени,
    на которых ищется решение.
\end{definition}

Стоит отметить, что определения~\ref{definition:classical_stiffness:linear_stiffness}
и \ref{definition:classical_stiffness:classical_stiffness}
фактически связывают жёсткость с сильно диссипативным характером системы,
вынуждающим либо уменьшать шаг интегрирования,
чтобы последний несколько раз укладывался в характерное время диссипации,
либо использовать неявные устойчивые методы,
которые численно воспроизводят эффект диссипации даже в случае больших шагов.
Для недиссипативных систем формальное определение жёсткости ввести труднее,
так как нет возможности достоверно отделить неустойчивое поведение численного метода
от неустойчивого поведения исходной системы.

% Рассмотренные определения не учитывают другие возможные эффекты,
% которые тоже могут считаться проявлениями жёсткости,
% так как вынуждают уменьшать шаг интегрирования.
% Поэтому далее по тексту жёсткость,
% введённая в определениях~\ref{definition:classical_stiffness:linear_stiffness}
% и \ref{definition:classical_stiffness:classical_stiffness},
% будет называться \emph{классической}.


\subsection{Экспоненциальные интеграторы}
\label{subsection:stiffness:exponential_integrators}

Линейный и нелинейный анализ устойчивости достаточно глубоко проработан.
Как следствие, в литературе предложено множество численных методов,
успешно справляющихся с эффектами
классической жёсткости~\cite{auzinger1993modern, dahlquist1963special, dahlquist1975stability, liu2019study, heirer1999solvingode2, lambert1991methods}.
В основном это различные многошаговые или многостадийные линейные схемы, обладающие теми или иными свойствами линейной устойчивости:
неявные методы Рунге-Кутты, методы Гира, формулы дифференцирования назад и прочее.
Обладая достаточной устойчивостью, они одновременно могут иметь высокий порядок аппроксимации.
Тем не менее, неявный характер данных методов может нивелировать все преимущества увеличенного шага интегрирования.

В настоящем разделе предлагается отойти от стандартных линейных схем с постоянными коэффициентами
и обратить внимание на численные методы,
утилизирующие информацию о матрице Якоби правой части дифференциального уравнения.
Понимание локального поведения системы может позволить частично или полностью устранить эффекты линейной жёсткости.
Более того, так как при решении жёстких систем обычно используются неявные методы,
и возникающие при этом алгебраические уравнения зачастую решаются методом Ньютона,
матрица Якоби правой части вычисляется в любом случае.
Поэтому её использование для модификации численной схемы не приносит дополнительной вычислительной сложности
(кроме, быть может, последующих операций с матрицей).

Методы, использующие матрицу Якоби правой части интегрируемой системы, называются \emph{адаптивными}.
Среди них можно перечислить методы Розенброка, Обрешкова, а также \emph{экспоненциальные интеграторы}.
В данной работе основное внимание будет уделено именно последним.

\begin{definition}
    \label{definition:exponential_integrators:exponential_integrator}
    Экспоненциальным интегратором называется численный метод,
    дающий применительно к линейной задаче~\eqref{eq:linear_stability_theory:linearized_initial_value_problem} точное решение.
\end{definition}

\begin{remark}
    \label{remark:exponential_integrators:exponential_integrator_stability_function}
    Любой экспоненциальный интегратор обладает функцией устойчивости $ R(z) = e^z $.
    Он также является L"=устойчивым.
\end{remark}

Из определения и замечания видно, что экспоненциальный интегратор точно интегрирует линейную часть системы.
Это позволяет исключить влияние линейной жёcткости на решение;
при этом численная схема вполне может быть полностью явной,
позволяя избавиться от необходимости решать алгебраические уравнения на каждом шаге интегрирования.
Преимуществом экспоненциальных интеграторов также является полное отсутствие численной
(то есть обусловленной исключительно методом, а не задачей) диссипации в линейных системах.
В некоторых задачах, близких к линейным, это качественно отличает данные методы от
стандартных A"=устойчивых схем с большими областями устойчивости
(пример можно увидеть в разделе \ref{sec:Lotka-Volterra}).

Некоторые авторы также расширяют определение экспоненциальных интеграторов на все методы,
использующие экспоненту матрицы Якоби правой части системы.
Обзор существующих экспоненциальных интеграторов можно найти, например,
в работах~\cite{minchev2005expint, hochbruck_ostermann_2010, hollevoet2013exponential_fitting}.
Мы же остановимся только на простейших из них, необходимых для дальнейшего анализа.

Рассмотрим явный и неявный экспоненциальные методы Эйлера:
%
\begin{align}
    %\text{\emph{Явный:}}   && \bvec{x}^{n+1} - \bvec{x}^n &= \left( \exp(\Delta t \cdot F) - \identity \right) F^{-1} \cdot f(\bvec{x}^n) \label{eq:exponential_integrators:ETDRK1_explicit} \\
    %\text{\emph{Неявный:}} && \bvec{x}^{n+1} - \bvec{x}^n &= \left( \identity - \exp(- \Delta t \cdot F) \right) F^{-1} \cdot f(\bvec{x}^{n+1}) \label{eq:exponential_integrators:ETDRK1_implicit}
    \text{\emph{Явный:}}   && \bvec{x}^{n+1} - \bvec{x}^n &= \varphi_1(\Delta t \cdot F) \cdot \Delta t \, f(\bvec{x}^n), \label{eq:exponential_integrators:ETDRK1_explicit} \\
    \text{\emph{Неявный:}} && \bvec{x}^{n+1} - \bvec{x}^n &= -\varphi_1(-\Delta t \cdot F) \cdot \Delta t \, f(\bvec{x}^{n+1}), \label{eq:exponential_integrators:ETDRK1_implicit}
\end{align}
%
где $ \varphi_1(z) $ определена в замечании~\ref{remark:linear_stability_theory:phi_1_function},
а $ F = \frac{\partial f}{\partial \bvec{x}} $~--- матрица Якоби правой части системы.
При этом точка, в которой вычисляются производные, может варьироваться.
Для линейных систем это не важно, но для нелинейных является определённой степенью свободы, заложенной в методе.
На самом деле, возможно даже использование приближённого значения матрицы,
но при этом экспоненциальный интегратор не будет давать точного решения на тестовой задаче Далквиста.

Несложно проверить, что при рассмотрении линеаризованной задачи
$ f(\bvec{x}) = f_0 + F_0 \cdot (\bvec{x} - \bvec{x}_0) $
оба метода дают точное решение~\eqref{eq:linear_stability_theory:linearized_solution}.
Следует отметить, что коэффициент при $ f(\blankarg) $ в обоих методах может расти экспоненциально с увеличением $ \Delta t $.
В таком случае с увеличением шага численное решение может быстро выйти за пределы окрестности,
в которой применима линеаризация правой части.
Это следует отнести к недостаткам данных методов,
так как такое поведение потенциально может являться источником неустойчивости в случае сильной нелинейности правой части.
В качестве одного из результатов работы также будет приведён двухточечный экспоненциальный интегратор,
построенный по аналогии с \eqref{eq:exponential_integrators:ETDRK1_explicit}
и \eqref{eq:exponential_integrators:ETDRK1_implicit},
но лишённый упомянутого недостатка.


\subsection{Нелинейная жёсткость}
\label{subsection:stiffness:nonlinear_stiffness}

В предыдущих разделах было показано, что с классической жёсткостью успешно справляются
устойчивые неявные методы, а также экспоненциальные интеграторы,
точно интегрирующие жёсткую часть системы в линейном приближении.
Практика, однако, показывает, что при решении систем с сильно нелинейной в том или ином смысле правой частью
могут возникать различные нежелательные эффекты неустойчивости,
даже если используются A"=устойчивые, L"=устойчивые или адаптивные методы.
Данные эффекты также вынуждают ограничивать величину шага.
Таким образом, система проявляет свойства,
которые невозможно объяснить в рамках классического понимания жёсткости.

Для примера рассмотрим задачу Коши
%
\begin{equation}
    \label{eq:nonlinear_stiffness:cosine_system}
    \begin{dcases}
        \frac{d x}{d t} = \cos\left( \frac{\pi}{2} \cdot x \right) \\
        x(0) = x_0 = 0
    \end{dcases}
    \qquad
    F(x) = \frac{\partial f}{\partial x}(x) = - \frac{\pi}{2} \sin\left( \frac{\pi}{2} \cdot x \right),
\end{equation}
%
имеющую точное решение
\begin{equation}
    \label{eq:nonlinear_stiffness:cosine_system_solution}
    x(t) = \frac{2}{\pi} \arcsin\left( \tanh\left( \frac{\pi}{2} t \right) \right)
\end{equation}

Рассматриваемая система является автономной и имеет множество положений равновесия $ x = 2k + 1, \; k \in \integers $.
Из них устойчивые только $ x = 4k + 1, \; k \in \integers $.
В окрестности каждого из положений равновесия система достаточно хорошо линеаризуема (с кубической точностью).
Более того, в любой точке функция $ f(x) $ отличается от свой линеаризации в ближайшем положении равновесия не более, чем на
$ |\cos(\pi/2) - \pi/2| \approx 1.58 \approx 1.58 \cdot \max |f(x)| $.
Наконец, в любой момент времени точное решение находится в промежутке $ [0; 1) $,
с экспоненциальной скоростью стремясь к положению равновесия $ x = 1 $.
Для $ x \in [0; 1) $ имеем $ -\pi /2 < F(x) \leqslant 0 $, то есть $ F(x) \in \complexes^- $.
Исходя из этого можно выдвинуть предположение, что численное решение данной системы
при помощи A"=устойчивого метода не вызвовет никаких проблем,
даже если взять шаг интегрирования, сравнимый с $ \min (1 / |F|) = 2 / \pi \sim \taulin $.

Результаты интегрирования рассматриваемой системы различными численными методами
приведены на рисунке~\ref{fig:nonlinear_stiffness:nonlinear_instability_example}.
Использовался шаг по времени $ \Delta t = 2 $.
В случае неявных схем возникающие при интегрировании нелинейные уравнения
решались методом Ньютона с начальным приближением в текущей точке.
Для сравнения также приведён график точного решения.

\begin{figure}[ht!]
    \centering
    \small
    \begin{gnuplot}[terminal=tikz, terminaloptions={color size 16cm,9.0cm fontscale 0.9}]
        load './gnuplot/common.gp'

        #set style increment default
        set style data linespoints
        set xlabel  "$ t $"
        set xrange  [ 0 : * ] noreverse writeback
        set ylabel  "$ x(t) $" #rotate by 0
        set yrange  [ -2 : 4 ] noreverse writeback

        set key width -32

        true_solution(x) = 2 * asin( tanh(pi * x / 2.0) ) / pi

        set xtics 2
        set ytics 1
        set xzeroaxis lw 2

        plot './data/exponential_integrators/tests.csv' every ::1 using 1:2 t 'явный (+явный экспоненциальный) метод Эйлера' ps 2, \
             './data/exponential_integrators/tests.csv' every ::1 using 1:3 t 'неявный (+неявный экспоненциальный) метод Эйлера' ps 2, \
             './data/exponential_integrators/tests.csv' every ::1 using 1:4 t 'метод трапеций' ps 2, \
             true_solution(x) t 'точное решение' with lines ls 200
    \end{gnuplot}
    \caption{Поведение простейших экспоненциальных интеграторов при решении уравнения~\eqref{eq:nonlinear_stiffness:cosine_system}}
    \label{fig:nonlinear_stiffness:nonlinear_instability_example}
\end{figure}

Как можно видеть, при использовании явного метода Эйлера наблюдаются незатухающие осцилляции
численного решения вокруг устойчивого положения равновесия $ x = 1 $,
что говорит о необходимости использования устойчивых методов.
Однако L"=устойчивый неявный метод Эйлера даёт совершенно некорректное решение,
уходящее в противоположную от аналитического решения сторону
и осциллирующее вокруг положения неустойчивого равновесия $ x = -1 $.
Использование экспоненциальных интеграторов из раздела~\ref{subsection:stiffness:exponential_integrators}
даёт идентичные результаты, так как в точках $ x = 2k, \; k \in \integers $ производная правой части вырождается.
Это означает, что наблюдаемое явление не объясняется только классической жёсткостью,
эффекты которой должны устраняться неявным методом Эйлера и экспоненциальными интеграторами.
С другой стороны, использование метода трапеций позволяет получить корректное и довольно точное численное решение.

Данный пример показывает, что явление жёсткости не ограничивается только
классическими определениями~\ref{definition:classical_stiffness:linear_stiffness}
и \ref{definition:classical_stiffness:classical_stiffness},
но также связано с нелинейностью правой части,
<<транслируемой>> неявными методами на дискретное уравнение.
Чтобы лучше проиллюстрировать это,
выпишем невязку и соответствующую матрицу Якоби тета-метода,
частными случаями которого являются рассмотренные явный и неявный методы Эйлера и метод трапеций:
%
\begin{equation}
    \label{eq:nonlinear_stiffness:theta-method_residual}
    \res(t^{n+1}, \bvec{x}^{n+1}) = \bvec{x}^{n+1} - \bvec{x}^n -
    \Delta t \cdot \left[ \theta f(t^{n+1}, \bvec{x}^{n+1}) + (1 - \theta) f(t^n, \bvec{x}^n) \right]
\end{equation}
%
\begin{equation}
    \label{eq:nonlinear_stiffness:theta-method_jacobian}
    \jac(t^{n+1}, \bvec{x}^{n+1}) = \frac{\partial \res}{\partial \bvec{x}}(t^{n+1}, \bvec{x}^{n+1}) =
    \identity - \Delta t \cdot \theta F(t^{n+1}, \bvec{x}^{n+1})
\end{equation}
%
В нашем случае $ f(x) $ и $ F(x) $ имеют целое семейство корней.
Поэтому при достаточно большом $ \Delta t $ и $ \theta \neq 0 $ несколько корней также имеет $ \res(x) $,
а $ \jac(x) $ перестаёт быть отделимым от нуля.
На рисунке~\ref{fig:nonlinear_stiffness:example_theta-method_residual}
показан график $ \res(x) $ для $ \Delta t = 2 $ и $ \theta \in \{0, 1/2, 1\} $;
также указано положение точного решения в момент времени $ t = \Delta t $.

\begin{figure}[ht!]
    \centering
    \small
    \begin{gnuplot}[terminal=tikz, terminaloptions={color size 16cm,6.0cm fontscale 0.9}]
        load './gnuplot/common.gp'

        set xlabel  "$ x $"
        set xrange  [ -4 : 4 ] noreverse writeback
        set ylabel  "$ \\res(x) $" #rotate by 0
        set yrange  [ -6 : 6 ] noreverse writeback

        set key bottom

        delta_t = 2.0

        f(x, theta) = x - delta_t * (theta * cos(0.5 * pi * x) + (1.0 - theta) * 1.0)
        f_0(x)  = f(x, 0.0)
        f_05(x) = f(x, 0.5)
        f_1(x)  = f(x, 1.0)
        true_solution(x) = 2 * asin( tanh(pi * x / 2.0) ) / pi

        set xtics 1
        set ytics 2
        set xzeroaxis lw 2
        set yzeroaxis lw 2

        xPos = true_solution(delta_t)
        set arrow 1 at xPos, graph 0 to xPos, graph 1 nohead ls 200
        set label 1 at xPos, graph 1 "$ x(\\Delta t) $" offset 0.5,-1.0

        plot f_0(x)  t "$ \\theta = 0   $" with lines, \
             f_05(x) t "$ \\theta = 1/2 $" with lines, \
             f_1(x)  t "$ \\theta = 1   $" with lines
    \end{gnuplot}
    \caption{График невязки~\eqref{eq:nonlinear_stiffness:theta-method_residual}
        при численном интегрировании системы~\eqref{eq:nonlinear_stiffness:cosine_system} с шагом $ \Delta t = 2 $.}
    \label{fig:nonlinear_stiffness:example_theta-method_residual}
\end{figure}

Видно, что $ \theta = 0 $ приводит к промаху мимо положения равновесия
(неустойчивое поведение явного метода Эйлера),
$ \theta = 1 $ порождает два дополнительных паразитических корня
(неявный метод Эйлера <<транслирует>> нелинейность $ f(x) $ на дискретизованную задачу),
а $ \theta = 1/2 $ позволяет добиться хорошего баланса между устойчивостью метода и линейностью уравнения $ \res(x) = 0 $.

Теорема Канторовича \cite{kantorovich1949method,ortega2000iterative}
даёт достаточные условия сходимости метода Ньютона к корню уравнения.
Одним из условий как раз является отделимость матрицы Якоби от нуля.
Таким образом, из-за нелинейности правой части и большого шага интегрирования, с одной стороны,
теряются достаточные условия сходимости метода Ньютона, и, с другой стороны, появляются некорректные корни невязки.
Приведённый пример показывает, что это может оказаться достаточным для того, чтобы метод выдавал некорректные решения.
Как показано в \cite{lambert1991methods}, более устойчивый метод простой итерации для поиска корней
также может расходиться в случае жёстких систем и большого шага интегрирования.
Можно предположить, что это универсальное свойство некоторых систем,
заставляющее делать выбор между величиной шага и сложностью алгоритма поиска корней нелинейной невязки.
%Можно предположить, что эта участь преследует все локальные
%(то есть не имеющие для широкого класса уравнений гарантированной возможности найти все корни)
%методы.

Таким образом, мы приходим к новому определению жёсткости системы,
связанному теперь с её нелинейным характером.

\begin{definition}
    \label{def:nonlinear_stiffness}
    Система вида $ \dot{\bvec{x}} = f(t, \bvec{x}) $ называется \emph{нелинейно жёсткой} в том случае,
    если нелинейные свойства правой части существенно влияют на
    %допустимую величину шага интегрирования,
    %при которой возможна сходимость используемого метода поиска корней невязки выбранной численной схемы,
    %либо на сложность используемого метода поиска корней,
    %способного для выбранного шага интегрирования корректно решать возникающие нелинейные уравнения.
    % \begin{itemize}
    %     \item
    %         допустимую величину шага интегрирования,
    %         при которой возможна сходимость используемого алгоритма поиска корней невязки выбранной численной схемы;
    %     \item
    %         допустимую степень неявности используемой численной схемы,
    %         при которой для заданного шага интегрирования возможна сходимость используемого алгоритма поиска корней невязки.
    %     \item
    %         сложность алгоритма поиска корней,
    %         достаточного для успешного нахождения нулей невязки выбранной численной схемы при фиксированном шаге интегрирования.
    % \end{itemize}
    \begin{itemize}[itemsep=0em]
        \item
            допустимую величину шага интегрирования для заданного алгоритма поиска корней невязки выбранной численной схемы;
        \item
            допустимую степень неявности используемой численной схемы для заданного шага интегрирования и алгоритма поиска корней невязки;
        \item
            сложность алгоритма поиска корней невязки для заданной численной схемы и шага интегрирования,
    \end{itemize}
    при которых возникающая на каждом шаге по времени нелинейная система алгебраических уравнений решается корректно.
\end{definition}

Иначе говоря, нелинейная жёсткость проявляется в необходимости выбирать между неявностью численной схемы,
большим шагом по времени и простотой метода решения нелинейных уравнений.

Таким образом, борьба с нелинейной жёсткостью переводится в плоскость методов оптимизации.
Среди способов улучшения сходимости метода Ньютона можно перечислить линейный поиск~\cite{armijo1966minimization, wolfe1969convergence},
метод доверительных областей \cite{sorensen1982newton} и различного рода ускорения~\cite{anderson1965iterative, nesterov27method, brown1994convergence}.
Линейный поиск минимизирует невязку вдоль выбранного направления путём подбора оптимального шага.
Метод доверительных областей изменяют направление шага, используя информацию о производных высшего порядка.
Ускоренные методы используют историю шагов при решении задачи оптимизации.
Возможна также комбинация упомянутых методов \cite{brune2015composing}.
Квазиньютоновские методы активно используются для решения уравнений, возникающих при интегрировании жёстких систем
\cite{brown1985experiments, alexander1991modified, moore1994stepsize, schlenkrich2006application}.
Данная группа методов решает задачу оптимизации или поиска корней уравнения,
используя аппроксимации производных, а не их точные значения.
Особо жёсткие задачи могут потребовать применения крайне медленных и непрактичных алгоритмов
полного поиска корней~\cite{farrell2016computation}.
Все эти методы отличаются необходимым количеством вычислений невязки, якобиана или гессиана в ходе поиска решения.
