\uchapter{Введение}
\label{chapter:introduction} \index{Введение}

Согласно отчётам Всемирной Организации Здравоохранения,
смерти от инсультов представляют для человечества серьёзную проблему,
занимая второе место по общемировой частоте,
уступая только смертям от сердечно-сосудистых заболеваний~\cite{geoffrey2008stroke, who2020global_health_estimates}.
Более того, даже в случае нелетального исхода инсульт является существенной угрозой здоровью человека
из-за сопутствующих осложнений.

Около восмидесяти процентов инсультов происходят из-за различных нарушений механизма тромбообразования~\cite{geoffrey2008stroke},
часто вызванных изменением биохимии крови,
геометрии отдельных участков сердечно-сосудистой системы
или установкой имплантов.
Это объясняет повышенный интерес к математическим моделям свёртывания крови,
позволяющим неинвазивно оценить риск образования тромба и разработать профилактические мероприятия.

Одним из главных вызовов в задаче моделирования тромбообразования является жёсткость реакций,
описывающих процесс свёртывания крови,
проявляющая себя даже в упрощённых, малокомпонентных моделях~\cite{bouchnita2020mathematical}.
Жёсткость каскада коагуляции выражается во <<взрывной>> динамике происходящих биохимических процессов,
в неустойчивости к малым возмущениям и в пороговом отклике на изменение параметров модели~\cite{shen2008threshold}.
Такое поведение системы вынуждает использовать малый шаг по времени даже при неявном численном интегрировании,
что приводит к непрактично большой длительности моделирования~\cite{douglas1967generalizedrk}.

Жёсткость, связанная с наличием разномасштабных процессов,
хорошо изучена в классических работах по устойчивости численных схем~\cite{auzinger1993modern, dahlquist1963special, dahlquist1975stability, liu2019study}.
Известно, что неявные методы численного интегрирования успешно подавляют нефизичные неустойчивости и осцилляции,
позволяя интегрировать жёсткие системы с разномасштабными процессами,
используя большой шаг по времени~\cite{heirer1999solvingode2}.
%дающая, однако, хорошие результаты для некоторых жёстких систем \cite{auzinger1989asymptotic}.
Малоизученной, однако, остаётся жёсткость, связанная с нелинейностью правой части системы дифференциальных уравнений,
хуже всего проявляющая себя как раз в случае неявных схем,
<<переносящих>> нелинейный характер системы на невязку дискретизованного уравнения;
это выражается в ухудшении или полном отсутствии сходимости традиционных методов поиска корней невязки~---
метода Ньютона и метода прямой итерации~\cite{lambert1991methods}.
Проблема нелинейной жёсткости частично решается использованием продвинутых
методов оптимизации~\cite{brown1985experiments, alexander1991modified, moore1994stepsize, schlenkrich2006application},
однако особо жёсткие задачи могут потребовать применения крайне медленных и непрактичных алгоритмов
полного поиска корней~\cite{farrell2016computation}.

Перспективным кажется подход, основанный на динамической интерполяции между явными и неявными численными схемами
для достижения баланса между устойчивостью и простотой нахождения корней.
В качестве примера можно привести замену компонент уравнений реакции,
дающих большой вклад во внедиагональную часть матрицы Якоби невязки,
на их экстраполированные значения~\cite{vassilevski2020parallel}.
Также ранее было предложено напрямую взвешивать явную и неявную часть метода Эйлера для
улучшения сходимости метода Ньютона при сохранении устойчивости численного решения~\cite{butakov2022two_methods}.

Другой важной задачей является построение согласованной модели образования фибринового (<<красного>>)
и тромбоцитарного (<<белого>>) тромба.
Традиционно выделяется два механизма коагуляции:
полимеризация фибрина, обычно происходящая в местах повреждения тканей~\cite{panteleev2008coagulation, ushakova2018gemo},
и слипание тромбоцитов в областях повышенного сдвигового напряжения~\cite{rahman2019platelet_adhesion, savage1996platelet_adhesion, avtaeva2022vWF}.
Первый механизм отличается сложностью каскада реакций свёртывания крови,
а второй~--- нетривиальным характером слипания тромбоцитов между собой и со стенками сосуда.
Также в обоих случаях образующийся тромб ведёт себя как взякопластичная и пористая среда,
что только усложняет моделирование.

В связи со сложностью полноценного математического описания каскада свёртывания крови
большинство моделей фокусируется лишь на отдельных аспектах коагуляции,
полностью или частично игнорируя другие существенные стороны процесса.
Зачастую воспроизводится образование либо фибринового тромба~\cite{bouchnita2020mathematical, vassilevski2020parallel},
либо тромбоцитарного~\cite{sorensen1999platelets_deposition_model, goodman2005thrombosis_model, taylor2016thrombosis_model, wu2017deposition_model}.
Это существенно сужает область применения имеющихся моделей.
Поэтому так важно получить наиболее общую,
но одновременно наиболее простую модель тромбообразования.

Настоящая работа посвящена как разработке новых методов интегрирования жёсткостких систем реакций,
так и созданию общей модели свёртывания крови.
В главе~\ref{chapter:theory} приводится краткий обзор традиционной теории жёстких систем дифференциальных уравнений,
обосновывается необходимость разделения жёсткости на <<классическую>>,
связанную с наличием в системе процессов, протекающих на разных временных масштабах,
и <<неклассическую>>, наблюдаемую в существенно нелинейных системах.
Глава~\ref{chapter:methods} посвящена развитию идеи динамической адаптации численных схем
для достижения баланса между устойчивостью и простотой поиска корней невязки дискретизованного уравнения.
В главе~\ref{chapter:experiments} проводятся испытания предложенных методов на наборе классических жётских задач.
Наконец, в главе~\ref{chapter:blood} приведены результаты разработки упрощённой модели образования белого тромба;
полученные уравнения планируется добавить в уже существующую модель роста
фибринового тромба~\cite{bouchnita2020mathematical, vassilevski2020parallel}.

% В работе также поднимается вопрос определения понятия жёсткости системы дифференциальных уравнений
% (отдельно обговорим, что будут рассматриваться только корректно поставленные задачи).
% С этой целью приведены основные положения линейной и нелинейной теории устойчивости, взятые из работ
% \cite{dahlquist1975stability, dahlquist1963special, lambert1991methods, heirer1999solvingode2}.
% В частности, рассмотрена обобщённая линейная задача Коши с ограниченным линейным оператором, действующим в банаховом пространстве.
% Приведены спектральные признаки устойчивости, а также формально доказан набор утверждений,
% связывающих область устойчивости численного метода и асимптотические свойства операторной экспоненты.
% Это позволяет ввести понятие линейной жёсткости и связать с ним линейную теорию устойчивости.
% Результаты нелинейной теории устойчивости позволяют обобщить это понятие на произвольные системы.
% В работе, однако, показано, что проблемы устойчивости интегрирования систем не всегда связаны только с линейной жёсткостью.
% Поэтому также предлагается ввести понятие нелинейной жёсткости,
% которое можно связать со сложностью оптимизационных задач,
% возникающих при использовании неявных численных методов.

% Согласно теореме Канторовича~\cite{kantorovich1949method,ortega2000iterative},
% для липшицевых в окрестности корня функций метод Ньютона локально сходится с квадратичной скоростью.
% В случае уравнений, возникающих при решении жёстких систем,
% односторонняя константа Липшица может оказаться сколь угодно большой,
% и сходимость ухудшается \cite{auzinger1990note, auzinger1993modern}.
% Среди методов по улучшению сходимости метода Ньютона можно перечислить линейный поиск \cite{armijo1966minimization, wolfe1969convergence},
% метод доверительных областей \cite{sorensen1982newton} и методы ускорения \cite{anderson1965iterative, nesterov27method, brown1994convergence}.
% Линейный поиск минимизирует невязку вдоль выбранного направления путём подбора оптимального шага.
% Метод доверительных областей изменяют направление шага, используя информацию о производных высшего порядка.
% Методы ускорения используют историю шагов при решении задачи оптимизации.
% Возможна также комбинация упомянутых методов \cite{brune2015composing}.
% Квазиньютоновские методы активно используются для решения уравнений, возникающих при интегрировании жёстких систем
% \cite{brown1985experiments, alexander1991modified, moore1994stepsize, schlenkrich2006application}.
% Данная группа методов решает задачу оптимизации или поиска корней уравнения, используя аппроксимации производных, а не их точные значения.
% Все эти методы отличаются необходимым количеством вычислений невязки, якобиана или гессиана в ходе поиска решения.
