\chapter{Модель образования белого тромба}
\label{chapter:blood} \index{Модель}

\section{Модели распределения тромбоцитов в объёме}
\label{section:volume_distribution_models}

%При моделировании слипания тромбоцитов и механических свойств слипшейся массы разумно отказаться от феноменологического подхода
%и получить необходимые законы из вспомогательных теоретических результатов.
%В перспективе это может позволить уменьшить число параметров модели,
%сделать их более интерпретируемыми.

%В основе слипания тромбоцитов и роста вязкости тромба лежит связывание тромбоцитов
%сетью из развёрнутых молекул фактора фон Виллебранда, длина которых 

Распределение тромбоцитов в объёме напрямую влияет на формирование связей между ними,
что сказывается как на скорости слипания тромбоцитов, так и на механических свойствах слипшейся массы.
Отсюда следует важность отдельного моделирования объёмного распределения тромбоцитов.

В данном разделе рассматривается вывод закона распределения из небольшого числа интуитивно понятных предположений.
Во-первых, предполагается, что при достижении предельной концентрации взаимное расположение тромбоцитов
близко к плотной упаковке трёхмерных шаров.
Во-вторых, считается, что при малых концентрациях тромбоциты распределены по объёму почти равномерно и случайно,
их взаимодействие между собой несущественно.
В-третьих, предполагается существование для средних значений концентрации переходного режима,
в котором распределение тромбоцитов всё еще достаточно случайное,
но взаимодействием между ними пренебрегать уже нельзя в силу значительности занимаемого ими объёма.


\subsection{Случай высокой концентрации}
\label{subsection:volume_distribution_models:high_concentration}

Высокая концентрация соответствует сформировавшемуся тромбу.
Тромбоциты в таком тромбе прилегают друг другу плотно.
В приближении сферически симметричных тромбоцитов
можно задействовать известный результат для плотной упаковки трёхмерных шаров:

\begin{theorem}[Гаусс, Кеплер, Хейлз]
     В пространстве $ \reals^3 $ рассмотрим последовательность ограниченных измеримых множеств $ S_n $ таких,
     что $ n $-ое множество содержит шар радиуса хотя бы $ n $.
     Пусть $ \setfamily_n $~--- семейство подмножеств $ S_n $,
     являющихся дизъюнктным объединением единичных шаров.
     Пусть $ \varphi \colon \bigcup_{n \in \naturals} \setfamily_n \to \naturals_0 $
     отображает дизъюнктное объединение шаров в число шаров в этом объединении.
     Тогда
     \[
         \lim_{n \to \infty} \max_{R \in \setfamily_n} \frac{\varphi(R)}{\mu(S_n)} = \frac{\pi}{3 \sqrt{2}}
     \]
\end{theorem}
Таким образом, максимально возможная концентрация частиц радиуса $ R $ равна
\[
    n_\textnormal{max} = \frac{\pi}{3 \sqrt{2}} \cdot \frac{1}{\frac{4}{3} \pi R^3}
    = \frac{1}{4 \sqrt{2} R^3}
    = \frac{\sqrt{2}}{(2R)^3}
\]
Размер тромбоцитов в человеческой крови составляет примерно $ 3 - 4 $ мкм.
Приняв этот размер за диаметр, получаем значение предельной концентрации:
от $ 2 \cdot 10^{16} \; \textnormal{м}^{-3} $ до $ 6 \cdot 10^{16} \; \textnormal{м}^{-3} $,
или примерно $ 40 \cdot 10^6 \; \textnormal{мкл}^{-1} $.
При нормальных условиях концентрация тромбоцитов в крови человека примерно равна $ 0.3 \cdot 10^6 \; \textnormal{мкл}^{-1} $,
что более чем в $ 100 $ раз меньше предельной концентрации
(или, иначе говоря, среднее расстояние между тромбоцитами примерно в $ \sqrt[3]{100} \approx 5 $ раз больше их радиуса).

Таким образом, можно вполне справедливо полагать,
что на ранних этапах образования тромба концентрация тромбоцитов достаточно мала.


\subsection{Случай низкой концентрации}
\label{subsection:volume_distribution_models:low_concentration}

Рассмотрим сосуд объёма $ V_0 $,
наполненный жидкостью с равномерно распределённой примесью концентрации $ n = N / V_0 $.
Число частиц в части сосуда объёма $ V $ распределено биномиально с параметром $ p = V n / N = V / V_0 $:
\[
    \proba\{ \textnormal{\# частиц} = k \} = \binom{N}{k} \left(\frac{V}{V_0}\right)^k \left(1 - \frac{V}{V_0} \right)^{N-k}
\]
При устремлении $ V_0 / V \to +\infty $ получаем известный результат~---
распределение Пуассона для числа частиц в малом объёме:
\begin{multline*}
    \proba\{ \textnormal{\# частиц} = k \} \sim \frac{N^k}{k!} \left( \frac{V}{V_0} \right)^k \left( 1 - \frac{V}{V_0} \right)^N = \\ =
    \frac{(V_0 n)^k}{k!} \left( \frac{V}{V_0} \right)^k \left( 1 - \frac{V}{V_0} \right)^{V_0 n} \sim \frac{(V n)^k}{k!} e^{- V n} 
\end{multline*}
Заметим, что вероятность обнаружить в объёме $ V $ не менее $ k $ частиц равна
\begin{multline}
    \label{eq:volume_distribution_models:volume_distribution}
    \proba\{ \textnormal{\# частиц} \geqslant k \} =
    1 - \proba\{ \textnormal{\# частиц} < k \} =
    1 - \sum_{m = 0}^{k-1} \proba\{ \textnormal{\# частиц} = m \} = \\ =
    1 - e^{- V n} \sum_{m = 0}^{k-1} \frac{(V n)^m}{m!}
\end{multline}
Полученное выражение является функцией распределения для
\emph{гамма-рас\-пре\-де\-ле\-ния} с параметрами $ k $, $ 1/n $.


Нас интересует вероятностное распределение расстояния от выбранной частицы до $ k $-ого ближайшего соседа
(обозначим это расстояние $ r_k $).
То есть, что эквивалентно, вероятностное распределение объёма шара,
в котором есть хотя бы $ k $ частиц (помимо центра).

Обозначим $ V(r) $ и $ S(r) $~--- объём шара и площадь сферы радиуса $ r $ соответственно
(в трёхмерном случае~--- $ \frac{4}{3} \pi r^3 $ и $ 4 \pi r^2 $ соответственно).
\begin{equation}
    \label{eq:volume_distribution_models:kNN-volume_equvalence}
    \proba\{r_k < x\} = \proba \left\{\textnormal{в $ V = V(x) $ находится $ \geqslant k $ частиц} \right\}
\end{equation}
Комбинируя~\eqref{eq:volume_distribution_models:volume_distribution} и \eqref{eq:volume_distribution_models:kNN-volume_equvalence},
получаем функцию распределения для $ r_k $:
\begin{equation}
    \label{eq:volume_distribution_models:kNN_distance_CDF}
    F_{r_k}(x) = 1 - e^{- V(x) n} \sum_{m = 0}^{k-1} \frac{(V(x) n)^m}{m!}
\end{equation}
Дифференцируя по $ x $ (с учётом соотношения $ \partial V(x) / \partial x = S(x) $),
получаем плотность распределения:
\begin{multline}
    \label{eq:volume_distribution_models:kNN_distance_PDF}
    \rho_{r_k}(x) = e^{-V(x) n} \cdot \left[ S(x) n \sum_{m = 0}^{k-1} \frac{(V(x) n)^m}{m!} -
    \sum_{m = 1}^{k-1} \frac{S(x) n (V(x) n)^{m-1}}{(m-1)!} \right] = \\ =
    S(x) n \cdot e^{-V(x) n} \frac{(V(x) n)^{k-1}}{(k-1)!}
\end{multline}
Характерный вид $ F_{r_k}(x) $ и $ \rho_{r_k}(x) $ представлен на графике~\ref{fig:volume_distribution_models:kNN_distance_distribution_plots}.

\begin{figure}[ht!]
    \centering
    \begin{subfigure}[t]{0.5\textwidth}
        \centering
        \captionsetup{aboveskip=-\baselineskip}
        \begin{gnuplot}[terminal=tikz, terminaloptions={color size 8.0cm,5.0cm fontscale 0.8}]
            load "./gnuplot/common.gp"
            load "./gnuplot/kNN_distribution.gp"

            set yrange [0:1]

            f_1(x)  = F_r(x,3,1)
            f_4(x)  = F_r(x,3,4)
            f_10(x) = F_r(x,3,10)

            plot [0:3] f_1(x) lw 2 t "$ F_{r_1}(x) $", \
                       f_4(x) lw 2 t "$ F_{r_4}(x) $", \
                       f_10(x) lw 2 t "$ F_{r_{10}}(x) $"
        \end{gnuplot}
        \caption{Функции распределения $ F_{r_k}(x) $.}
    \end{subfigure}%
    \begin{subfigure}[t]{0.5\textwidth}
        \centering
        \captionsetup{aboveskip=-\baselineskip}
        \begin{gnuplot}[terminal=tikz, terminaloptions={color size 8.0cm,5.0cm fontscale 0.8}]
            load "./gnuplot/common.gp"
            load "./gnuplot/kNN_distribution.gp"

            f_1(x)  = rho_r(x,3,1)
            f_4(x)  = rho_r(x,3,4)
            f_10(x) = rho_r(x,3,10)

            plot [0:3] f_1(x) lw 2 t "$ \\rho_{r_1}(x) $", \
                        f_4(x) lw 2 t "$ \\rho_{r_4}(x) $", \
                        f_10(x) lw 2 t "$ \\rho_{r_{10}}(x) $"
        \end{gnuplot}
        \caption{Функций плотности вероятности $ \rho_{r_k}(x) $.}
    \end{subfigure}
    \caption{Графики распределения случайной величины $ r_k $ для $ k \in \{1, 4, 10 \} $.}
    \label{fig:volume_distribution_models:kNN_distance_distribution_plots}
\end{figure}


% В общем случае объём и площадь поверхности $ d $-мерного шара равны, соответственно,
% \begin{equation}
%     \label{eq:volume_distribution_models:d-dim_ball_volume_and_surface}
%     V_d(r) = \frac{\pi^{d/2}}{\Gamma(d/2 + 1)} \cdot r^d \qquad S_d(r) = 2 \pi V_{d-1}(r)
% \end{equation}
% Отсюда имеем
% \[
%     r = \frac{1}{\sqrt{\pi}} \left[ \Gamma(d/2+1) V_d(r) \right]^{1/d},
% \]
Найдём моменты случайной величины $ r_k $.
Вспомним, что объём $ d $-мерного шара радиуса $ r $ пропорционален $ r^d $:
\[
    V_d(r) = V_d(1) \cdot r^d \quad \Longrightarrow \quad r = \sqrt[d]{V_d(r) / V_d(1)},
\]
Тогда
\begin{align}
    \label{eq:volume_distribution_models:kNN_distance_expectation}
    \expect r_k^m &= \int\limits_0^{+\infty} x^m \rho_{r_k}(x) \, dx =
    \int\limits_0^{+\infty} x^m e^{-V(x) n} \frac{(V(x) n)^{k-1}}{(k-1)!} \, S(x) n \, dx = \\
    &= \int\limits_0^{+\infty} x^m(V) \cdot e^{-V n} \frac{(V n)^{k-1}}{(k-1)!} \, d(V n) = \\
    &= V(1)^{-m/d} \int\limits_0^{+\infty} V^{m/d} \cdot e^{-V n} \frac{(V n)^{k-1}}{(k-1)!} \, d(V n) = \\
    &= \frac{(V(1) n)^{-m/d}}{(k-1)!} \int\limits_0^{+\infty} z^{m/d + k - 1} e^{-z} dz
    = \frac{\Gamma(m/d + k)}{\Gamma(k)} (V(1) n)^{-m/d}
\end{align}
В частности, в трёхмерном случае среднее расстояние до ближайшего соседа примерно равно $ 0.83 \cdot n^{-1/3} $.

% \begin{figure}
%     \centering
%     \begin{subfigure}[t]{0.5\textwidth}
%         \centering
%         \begin{gnuplot}[terminal=tikz, terminaloptions={color size 8.5cm,5.0cm}]
%             load "./gnuplot/common.gp"
%             load "./gnuplot/kNN_distribution.gp"
% 
%             set yrange [0:1]
% 
%             f_1(x)  = F_r_normalized(x,3,1)
%             f_4(x)  = F_r_normalized(x,3,4)
%             f_10(x) = F_r_normalized(x,3,10)
% 
%             plot [-3:3] f_1(x) lw 2 t "$ F_{r_1}(x) $", \
%                         f_4(x) lw 2 t "$ F_{r_4}(x) $", \
%                         f_10(x) lw 2 t "$ F_{r_{10}}(x) $"
%         \end{gnuplot}
%         \caption{Характерный вид функций распределения $ F_{r_k}(x) $.}
%     \end{subfigure}%
%     \begin{subfigure}[t]{0.5\textwidth}
%         \centering
%         \begin{gnuplot}[terminal=tikz, terminaloptions={color size 8.5cm,5.0cm}]
%             load "./gnuplot/common.gp"
%             load "./gnuplot/kNN_distribution.gp"
% 
%             f_1(x)  = rho_r_normalized(x,3,1)
%             f_4(x)  = rho_r_normalized(x,3,4)
%             f_10(x) = rho_r_normalized(x,3,10)
% 
%             plot [-3:3] f_1(x) lw 2 t "$ \\rho_{r_1}(x) $", \
%                         f_4(x) lw 2 t "$ \\rho_{r_4}(x) $", \
%                         f_10(x) lw 2 t "$ \\rho_{r_{10}}(x) $"
%         \end{gnuplot}
%         \caption{Характерный вид плотности распределения $ \rho_{r_k}(x) $.}
%     \end{subfigure}
% \end{figure}
