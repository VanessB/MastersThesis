\uchapter{Заключение}
\label{chapter:summary} \index{Заключение}

Несмотря на бурное развитие теории устойчивости численных методов во второй половине прошлого века,
некоторые вопросы данной области и по сей день остаются без ответа.
В частности, до сих пор нет удовлетворительного определения жёстких систем,
даже несмотря на то, что необходимость их решения возникает повсеместно.
Например, уравнения, описывающие химические реакции,
могут быть особенно жёсткими в силу разных временных масштабов протекающих процессов, седлового характера,
а также в силу значительной нелинейности правой части.
Возможность численно решать подобные системы в условиях ограниченных вычислительных ресурсов
требует развития теории жёстких систем дифференциальных уравнений,
а также соответствующих вычислительных методов.

Проведённое исследование по большей части было посвящено указанным проблемам численного анализа.
Основные результаты обозначенной части работы приведены ниже:
\begin{enumerate}
    \item
        Систематизированы основные положения теории устойчивости численных методов,
        необходимые для исследования жёсткости систем.
        Введены два новых понятия: \emph{классическая} и \emph{нелинейная жёсткость},
        отражающие разную природу жёсткости различных систем.
    \item
        Показано, что для устранения эффектов классической жёсткости достаточно использовать
        устойчивые в том или ином смысле численные методы.
    \item
        Продемонстрировано, что понятие нелинейной жёсткости осмысленно и информативно:
        существуют жёсткие системы, которые некорректно интегрируются A"=устойчивыми,
        L"=устойчивыми методами и экспоненциальными интеграторами.
        Причём сложность интегрирования данных систем обусловлена нелинейностью правых частей,
        порождающей паразитические корни невязки дискретизованного уравнения.
        Более того, показано, что использование менее устойчивых численных методов
        может частично или полностью решить проблему нелинейной жёсткости для конкретных задач.
    \item
        Рассмотрен новый подход к построению численных схем,
        обобщающий метод экспоненциальной подгонки.
        Предложенный способ позволяет динамически адаптировать численные схемы
        для получения более точных и устойчивых численных решений определённого класса задач.
        Данный подход также позволяет соблюдать динамический баланс между устойчивостью численной схемы
        и простотой поиска корней невязки дискретизованного уравнения,
        что полезно при интегрировании жёстких во всех смыслах систем.
    \item
        Предложенный метод адаптации был применён ко множеству задач,
        встречающихся в том или ином виде в биологии, химии, экологии, экономике и небесной механике.
    \item
        На основе предложенного метода адаптации была предложена модификация метода Ньютона,
        позволяющая в случае сильно неявных численных методов улучшить сходимость ньютоновских итераций.
    \item
        Для разработанных методов проведены численные эксперименты на жёстких системах
        дифференциальных уравнений (в том числе и на системе каскада свёртывания крови),
        показывающие преимущество предложенного подхода в сравнении с другими классическими методами
        в вопросах борьбы с нежелательными эффектами нелинейной жёсткости.
    % \item
    %     Показан основной недостаток предложенных методов:
    %     большое число ньютоновских итераций, требуемых для поиска нулей невязки.
    %     Данный недостаток вызван квазиньютоновским характером методов.
    %     Возможные пути его устранения: <<заморозка>> весовой матрицы на все или несколько ньютоновских итераций,
    %     а также учет её производных.
\end{enumerate}

Заключительная глава работы посвящена разработке модификации существующей модели фибринового тромба,
которая должна позволить моделировать образование тромбоцитарного тромба.
Основные результаты данной части работы приведены ниже:
\begin{enumerate}
    \item
        Рассмотрены вопросы распределения тромбоцитов в объёме.
        Полученные теоретические результаты использованы для построения моделей вязкости
        и послойного роста тромба.
    \item
        Предложены модификации к системе уравнений реакций,
        описывающие слипание и смыв тромбоцитов.
    \item
        На основе существующих работ предложенные уравнения реакций были конкретизированы.
        Также был произведён начальный подбор большинства параметров модели.
\end{enumerate}

В дальнейшем необходимо провести валидацию модели и точный подбор параметров на существующих экспериментах с белым тромбом.
Итоговой целью является интеграция полученных уравнений в существующую модель фибринового тромба.

% В главе \ref{chapter:perspectives} также были описаны перспективы клинического применения предложенных методов
% для моделирования образования тромба в ушке левого предсердия,
% в том числе приведены планируемые направления развития.
% В частности, на основе данных методов планируется реализовать модуль для интегрирования реакционных частей
% систем переноса-диффузии-реакции.
