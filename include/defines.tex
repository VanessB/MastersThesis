%% Математические символы и прочие дефайны.

\def\defarr{\overset{\triangle}{\Longleftrightarrow}} % <<По определению>>
\def\defeq{\triangleq}                                % <<По определению равно>>
\def\symdiff{\,\triangle\,}                           % <<Симметрическая разность>>
\def\connected{\leftrightsquigarrow}                  % Связность в графах.
\def\optimal{\star}                                   % Оптимальное значение.

% Большая черта в множествах.
\def\Mid{\;\middle|\;}

% Стрелки для сходимостей.
\newcommand{\limarrow}[2][\longrightarrow]{\underset{#2}{#1}}
\newcommand{\limto}[2]{\limarrow{#1 \to #2}}


% Математическое операторы.
\DeclareMathOperator{\diam}{\textnormal{diam}}
\DeclareMathOperator{\rad}{\textnormal{rad}}
\DeclareMathOperator*{\argmin}{\arg\min}
\DeclareMathOperator*{\argmax}{\arg\max}
\DeclareMathOperator*{\dom}{\textnormal{dom}}
\DeclareMathOperator*{\range}{\textnormal{range}}
\DeclareMathOperator*{\closure}{\textnormal{cl}}
\DeclareMathOperator*{\lowlim}{\underline{lim}}
\DeclareMathOperator*{\uplim}{\overline{lim}}

\newcommand{\mean}[1]{\left\langle #1 \right\rangle} % Усреднение.

% Математические множества.
\def\naturals{\mathbb{N}}  % Натуральные числа.
\def\integers{\mathbb{Z}}  % Целые числа.
\def\rationals{\mathbb{Q}} % Рациональные числа.
\def\reals{\mathbb{R}}     % Действительные числа.
\def\complexes{\mathbb{C}} % Комплексные числа.

%% Функции
\def\blankarg{\,\cdot\,}

%% Линейная алгебра.
\def\diag{\operatorname{diag}} % Диагональная матрица.

%% Функциональный анализ.
\def\banachspace{\mathcal{B}}                                  % Банахово пространство.
\def\hilbertspace{\mathcal{H}}                                 % Гильбертово пространство.
\def\linopset{\mathcal{L}}                                     % Множество линейных ограниченных операторов.
\newcommand{\dotprod}[2]{\left\langle #1, #2 \right\rangle}    % Скалярное произведение.
\def\identity{\mathrm{I}}                                      % Единичный оператор.

%% Комплексные числа.
\def\Re{\operatorname{Re}} % Действительная часть.
\def\Im{\operatorname{Im}} % Мнимая часть.

%% Асимптотические классы.
\def\Oclass{\mathcal{O}} % О-большое.

%% Выделение определения.
\DeclareTextFontCommand{\defemph}{\bfseries\em}
