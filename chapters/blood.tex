\chapter{Моделирование образования тромба}
\label{chapter:blood} \index{Модель}

Данный раздел посвящен вопросам разработки универсальной модели коагуляции.
Важность медицинских приложений такой модели нельзя переоценить,
так как нарушения гемостаза могут вызывать серьезные осложнения
и являются одной из наиболее частых причин смерти~\cite{geoffrey2008stroke, jackson2011arterial_thrombosis, who2020global_health_estimates}.
Персонифицированные модели свёртывания крови в перспективе позволят выявлять пациентов из группы риска
и подбирать им индивидуальное лечение с минимальным вмешательством в механизмы гемостаза.
Подобные модели полезны и для анализа рисков при установке имплантов,
которые, как известно, могут как катализировать, так и подавлять свёртывание крови%
~\cite{rasche2001haemostasis_overview, bluestein2002mechanical_heart_valves_thrombosis, slepian2017shear-induced_platelet_activation, consolo2019shear-mediated_platelet_activation}.

%Как и многие другие биохимические системы,
%каскад коагуляции тяжело поддаётся математическому и компьютерному моделированию,
%о чём свидетельствует значительное число моделей разной степени сложности и полноты описания.
%\cite{sorensen1999platelets_deposition_model, goodman2005thrombosis_model, wufsus2013clot_permeability, taylor2016thrombosis_model, wu2017deposition_model, bouchnita2020mathematical, vassilevski2020parallel}.

Тромбы, значительный объем которых составляет полимеризованная масса фибрина,
называют \emph{фибриновыми}, а тромбы, в основном состоящие из слипшихся тромбоцитов~--- \emph{тромбоцитарными}.
Традиционно модели фокусируются на воспроизведении либо фибриновго тромба, либо тромбоцитарного.
С одной стороны, это обусловлено высокой сложностью каскада реакций полимеризации фибрина~\cite{rasche2001haemostasis_overview, bouchnita2020mathematical},
с другой~--- многообразием механизмов слипания тромбоцитов в артериальных потоках крови~\cite{savage1996platelet_adhesion, rasche2001haemostasis_overview, rahman2019platelet_adhesion}.
В основе разрабатываемой универсальной модели коагуляции лежит модель фибринового тромба,
предложенная в работах~\cite{bouchnita2020mathematical, vassilevski2020parallel}.
Поэтому основной упор будет сделан на модификации,
позволяющие моделировать рост тромбоцитарного тромба.

В первой секции настоящей главы приведён краткий обзор актуальных задач и характерных трудностей,
возникающих при моделировании процесса образования тромба.
%Вторая секция посвящена математической модели фибринового тромба,
%взятой за основу для дальнейшего улучшения в рамках данной работы.
Во второй секции поставлены и решены вспомогательные задачи,
необходимые для моделирования взаимодействия распределённых в объёме тромбоцитов.
Наконец, в третьей секции подробно описана модель тромбоцитарного тромба,
предложенная в рамках данной работы.


\section{Обзор актуальных задач и проблем}
\label{section:problems_overview}

При моделировании свёртывания крови приходится решать две существенные проблемы,
напрямую вытекающие из сложности и многообразия механизмов поддержания гемостаза.
Во-первых, каскад реакций свёртывания крови является жёсткой системой~\cite{kastrup2007threshold, shen2008threshold, butakov2022two_methods},
что вынуждает либо тратить значительные вычислительные ресурсы на интегрирование модели с малым шагом по времени,
либо использовать дорогостоящие продвинутые численные методы для обхода ограничений,
вызванных жёсткостью~\cite{vassilevski2020parallel, butakov2022two_methods}.
Жёсткое поведение является характерным свойством исходной системы,
позволяющим эффективно поддерживать гемостаз~\cite{kastrup2007threshold, shen2008threshold}.
Поэтому проблему жёсткости невозможно решить выбором <<нежёсткой>> модели,
не теряя при этом описательной способности.
Иначе говоря, формальное описание и решение указанной проблемы лежит в плоскости численных методов,
а потому оно вынесено в отдельные главы~\ref{chapter:theory} и~\ref{chapter:methods}.

Во-вторых, свёртывание крови может инициироваться, поддерживаться и подавляться механизмами,
значительно отличающимися друг от друга как по принципу работы, так и по роли в общем каскаде коагуляции%
~\cite{rasche2001haemostasis_overview},
что делает поиск компромисса между универсальностью и простотой модели крайне тяжелым.
Хорошим показателем высокой сложности построения универсальной модели свёртывания крови
служит тот факт, что процесс коагуляции может быть инициирован как химическим воздействием,
так и механическим.
Конкретно, образование тромба может быть вызвано длинной цепочкой биохимических реакций,
инициированной высвобождением тканевого фактора из повреждённого участка сосуда
и приводящей к полимеризации фибрина~\cite{rasche2001haemostasis_overview, panteleev2008coagulation, ushakova2018gemo}.
Также рост тромба может быть запущен слипанием тромбоцитов под действием коллагена~\cite{rasche2001haemostasis_overview, rahman2019platelet_adhesion},
фибрина, фибриногена~\cite{savage1996platelet_adhesion, rahman2019platelet_adhesion}
и фактора фон Виллебранда (vWF)~\cite{savage1996platelet_adhesion, rahman2019platelet_adhesion, avtaeva2022vWF}.
Причём в последнем случае наблюдается усиление слипания с ростом сдвиговых напряжений в потоке крови~\cite{savage1996platelet_adhesion, rahman2019platelet_adhesion}.
Этот эффект связывают с механическими свойствами фактора фон Виллебранда:
при малых скоростях сдвига фактор свёрнут в глобулу
и не может сцеплять тромбоциты между собой и со стенкой,
однако высокое сдвиговое напряжение позволяют глобулам развернуться,
что приводит к агрегации тромбоцитов~\cite{lippok2016vWF_unfolding, schneider2007vWF_unfolding, zhussupbekov2021vWF_unfolding}.

Несмотря на то, что современные исследования позволяют говорить о
решающей роли фактора фон Виллебранда в агрегации тромбоцитов~\cite{lauren2015high_shear_rate_thrombosis, mereuta2021white_clots},
построение полноценной модели слипания в значительной степени затруднено.
Поскольку фактор фон Виллебранда в свёрнутом состоянии не формирует связи с тромбоцитами~\cite{avtaeva2022vWF},
существенная часть исследований фокусируется на моделировании разворачивания фактора в быстрых потоках крови.
Здесь стоит упомянуть феноменологические модели%на основе \emph{логистической функции}
~\cite{lippok2016vWF_unfolding, schneider2007vWF_unfolding},
статическую~\cite{zlobina2016vWF_unfolding} и динамическую~\cite{pushin2020vWF_unfolding} модели разворачивания фактора с закреплённым концом,
а также модель разворачивания свободной молекулы,
учитывающую, помимо высокого сдвигового напряжения, другие механические факторы~\cite{zhussupbekov2021vWF_unfolding}.
Опыты, однако, показывают, что агрегация тромбоцитов начинается при сдвиговых напряжениях,
недостаточных для полного разворачивания фактора фон Виллебранда~\cite{savage1996platelet_adhesion, rahman2019platelet_adhesion}.
Таким образом, вышеописанные модели разворачивания фактора невозможно напрямую использовать для регуляции
слипания тромбоцитов в моделях тромбоцитарного тромба.
Наконец, в универсальной модели каскада коагуляции также необходимо учитывать не только агрегацию,
но и активацию тромбоцитов в потоках крови с высокими сдвиговыми напряжениями~\cite{lauren2015high_shear_rate_thrombosis, rasche2001haemostasis_overview}.

Отдельным вызовом также является моделирование механических свойств тромба.
Известно, что полимеризованный фибрин и слипшиеся тромбоциты образуют пористую среду,
что необходимо учитывать в виде переменного коэффициента проницаемости~\cite{wufsus2013clot_permeability}.
Также тромбоциты, скреплённые фактором фон Виллебранда, обладают сравнительно высокой подвижностью~\cite{savage1996platelet_adhesion},
что делает тромбоцитарный тромб вязкопластичной средой~\cite{jamiolkowski2016visualization}.


% \section{Модель фибринового тромба}
% \label{section:red_clot_model}
% 
% Одной из задач данной работы является разработка модификации уже существующей модели фибринового тромба,
% описанной в работах~\cite{bouchnita2020mathematical, vassilevski2020parallel}.
% Для полного понимания полученных результатов в настоящем разделе приводится сжатое описание данной модели.
% Модель основана на уравнении переноса-диффузии-реакции с упрощённым каскадом свёртывания крови.
% Она успешно воспроизводит эксперименты с ростом тромба в повреждённых сосудах,
% однако никак не учитывает слипание тромбоцитов в потоках с высокими сдвиговыми напряжениями,
% не позволяя моделировать образование тромбоцитарного тромба.


\section{Модель распределения тромбоцитов в объёме}
\label{section:volume_distribution_models}

Распределение тромбоцитов в объёме напрямую влияет на формирование связей между ними,
что сказывается как на механических свойствах слипшейся массы, так и на скорости слипания.
Отсюда следует важность отдельного моделирования объёмного распределения тромбоцитов.

В данном разделе рассматривается вывод закона распределения из небольшого числа интуитивно понятных предположений.
Во-первых, предполагается, что при достижении предельной концентрации взаимное расположение тромбоцитов
близко к плотной упаковке трёхмерных шаров.
Во-вторых, считается, что при малых концентрациях тромбоциты распределены по объёму почти равномерно и случайно,
их взаимодействие между собой несущественно.
В-третьих, предполагается существование для средних значений концентрации переходного режима,
в котором распределение тромбоцитов всё еще достаточно случайное,
но взаимодействием между ними пренебрегать уже нельзя в силу значительности занимаемого ими объёма.


\subsection{Случай высокой концентрации}
\label{subsection:volume_distribution_models:high_concentration}

Высокая концентрация соответствует сформировавшемуся тромбу.
Тромбоциты в таком тромбе прилегают друг другу плотно.
В приближении сферически симметричных тромбоцитов
можно задействовать известный результат для плотной упаковки трёхмерных шаров:

\begin{theorem}[Гаусс, Кеплер, Хейлз~\cite{hales2002overview_kepler}]
    \label{theorem:high_concentration:packed_balls}
    В пространстве $ \reals^3 $ рассмотрим последовательность ограниченных измеримых множеств $ S_n $ таких,
    что $ n $-ое множество содержит шар радиуса хотя бы $ n $.
    Пусть $ \setfamily_n $~--- семейство подмножеств $ S_n $,
    являющихся дизъюнктным объединением единичных шаров.
    Пусть $ \varphi \colon \bigcup_{n \in \naturals} \setfamily_n \to \naturals_0 $
    отображает дизъюнктное объединение шаров в число шаров в этом объединении.
    Тогда
    \[
        \lim_{n \to \infty} \max_{R \in \setfamily_n} \frac{\varphi(R)}{\mu(S_n)} = \frac{\pi}{3 \sqrt{2}}
    \]
\end{theorem}

Таким образом, максимально возможная концентрация частиц радиуса $ R $ равна
\[
    n_\textnormal{max} = \frac{\pi}{3 \sqrt{2}} \cdot \frac{1}{\frac{4}{3} \pi R^3}
    = \frac{1}{4 \sqrt{2} R^3}
    = \frac{\sqrt{2}}{(2R)^3}
\]
Размер тромбоцитов в человеческой крови составляет примерно
$ 3 - 4 $ мкм~\cite{rumbaut2010platelets-vessel_interactions}.
Приняв этот размер за диаметр, получаем значение предельной концентрации:
от $ 2 \cdot 10^{16} \; \textnormal{м}^{-3} $ до $ 6 \cdot 10^{16} \; \textnormal{м}^{-3} $,
или примерно $ 40 \cdot 10^6 \; \textnormal{мкл}^{-1} $.
При нормальных условиях концентрация тромбоцитов в крови человека примерно равна
$ 0.3 \cdot 10^6 \; \textnormal{мкл}^{-1} $~\cite{zhang2022COPD_risk},
что более чем в $ 100 $ раз меньше предельной концентрации
(или, иначе говоря, среднее расстояние между тромбоцитами примерно в $ \sqrt[3]{100} \approx 5 $ раз больше их радиуса).

Таким образом, можно вполне справедливо полагать,
что на ранних этапах образования тромба концентрация тромбоцитов достаточно мала.


\subsection{Случай низкой концентрации}
\label{subsection:volume_distribution_models:low_concentration}

Рассмотрим сосуд объёма $ V_0 $,
наполненный жидкостью с равномерно распределённой примесью концентрации $ n = N / V_0 $.
Число частиц в части сосуда объёма $ V $ распределено биномиально с параметром $ p = V n / N = V / V_0 $:
\[
    \proba\{ \textnormal{\# частиц} = k \} = \binom{N}{k} \left(\frac{V}{V_0}\right)^k \left(1 - \frac{V}{V_0} \right)^{N-k}
\]
При устремлении $ V_0 / V \to +\infty $ получаем известный результат~---
распределение Пуассона для числа частиц в малом объёме:
%
\begin{multline*}
    \proba\{ \textnormal{\# частиц} = k \} \sim \frac{N^k}{k!} \left( \frac{V}{V_0} \right)^k \left( 1 - \frac{V}{V_0} \right)^N = \\ =
    \frac{(V_0 n)^k}{k!} \left( \frac{V}{V_0} \right)^k \left( 1 - \frac{V}{V_0} \right)^{V_0 n} \sim \frac{(V n)^k}{k!} e^{- V n} 
\end{multline*}
%
Заметим, что вероятность обнаружить в объёме $ V $ не менее $ k $ частиц равна
%
\begin{multline}
    \label{eq:low_concentration:volume_distribution}
    \proba\{ \textnormal{\# частиц} \geqslant k \} =
    1 - \proba\{ \textnormal{\# частиц} < k \} =
    1 - \sum_{m = 0}^{k-1} \proba\{ \textnormal{\# частиц} = m \} = \\ =
    1 - e^{- V n} \sum_{m = 0}^{k-1} \frac{(V n)^m}{m!}
\end{multline}
%
Полученное выражение является функцией распределения для
\emph{гамма"=распределения} с параметрами $ k $, $ 1/n $.

Нас интересует вероятностное распределение расстояния от выбранной частицы до $ k $-ого ближайшего соседа
(обозначим это расстояние $ r_k $).
То есть, что эквивалентно, вероятностное распределение объёма шара,
в котором есть хотя бы $ k $ частиц (помимо центра).

Обозначим $ V(r) $ и $ S(r) $~--- объём шара и площадь сферы радиуса $ r $ соответственно
(в трёхмерном случае~--- $ \frac{4}{3} \pi r^3 $ и $ 4 \pi r^2 $ соответственно).
%
\begin{equation}
    \label{eq:low_concentration:kNN-volume_equvalence}
    \proba\{r_k < x\} = \proba \left\{\textnormal{в $ V = V(x) $ находится $ \geqslant k $ частиц} \right\}
\end{equation}
%
Комбинируя~\eqref{eq:low_concentration:volume_distribution} и \eqref{eq:low_concentration:kNN-volume_equvalence},
получаем функцию распределения для $ r_k $:
%
\begin{equation}
    \label{eq:low_concentration:kNN_distance_CDF}
    F_{r_k}(x) = 1 - e^{- V(x) n} \sum_{m = 0}^{k-1} \frac{(V(x) n)^m}{m!}
\end{equation}
%
Дифференцируя по $ x $ (с учётом соотношения $ \partial V(x) / \partial x = S(x) $),
получаем плотность распределения:
%
\begin{multline}
    \label{eq:low_concentration:kNN_distance_PDF}
    \rho_{r_k}(x) = e^{-V(x) n} \cdot \left[ S(x) n \sum_{m = 0}^{k-1} \frac{(V(x) n)^m}{m!} -
    \sum_{m = 1}^{k-1} \frac{S(x) n (V(x) n)^{m-1}}{(m-1)!} \right] = \\ =
    S(x) n \cdot e^{-V(x) n} \frac{(V(x) n)^{k-1}}{(k-1)!}
\end{multline}
%
Характерный вид $ F_{r_k}(x) $ и $ \rho_{r_k}(x) $ представлен
на рисунке~\ref{fig:low_concentration:kNN_distance_distribution_plots}.

\begin{figure}[ht!]
    \centering
    \begin{subfigure}[t]{0.5\textwidth}
        \centering
        \captionsetup{aboveskip=-\baselineskip}
        \begin{gnuplot}[terminal=tikz, terminaloptions={color size 8.0cm,5.0cm fontscale 0.8}]
            load "./gnuplot/common.gp"
            load "./gnuplot/kNN_distribution.gp"

            set yrange [0:1]

            f_1(x)  = F_r(x,3,1)
            f_4(x)  = F_r(x,3,4)
            f_10(x) = F_r(x,3,10)

            plot [0:3] f_1(x) lw 2 t "$ F_{r_1}(x) $", \
                       f_4(x) lw 2 t "$ F_{r_4}(x) $", \
                       f_10(x) lw 2 t "$ F_{r_{10}}(x) $"
        \end{gnuplot}
        \caption{Функции распределения $ F_{r_k}(x) $.}
    \end{subfigure}%
    \begin{subfigure}[t]{0.5\textwidth}
        \centering
        \captionsetup{aboveskip=-\baselineskip}
        \begin{gnuplot}[terminal=tikz, terminaloptions={color size 8.0cm,5.0cm fontscale 0.8}]
            load "./gnuplot/common.gp"
            load "./gnuplot/kNN_distribution.gp"

            f_1(x)  = rho_r(x,3,1)
            f_4(x)  = rho_r(x,3,4)
            f_10(x) = rho_r(x,3,10)

            plot [0:3] f_1(x) lw 2 t "$ \\rho_{r_1}(x) $", \
                        f_4(x) lw 2 t "$ \\rho_{r_4}(x) $", \
                        f_10(x) lw 2 t "$ \\rho_{r_{10}}(x) $"
        \end{gnuplot}
        \caption{Функций плотности вероятности $ \rho_{r_k}(x) $.}
    \end{subfigure}
    \caption{Графики распределения случайной величины $ r_k $ для $ k \in \{1, 4, 10 \} $.}
    \label{fig:low_concentration:kNN_distance_distribution_plots}
\end{figure}


% В общем случае объём и площадь поверхности $ d $-мерного шара равны, соответственно,
% \begin{equation}
%     \label{eq:volume_distribution_models:d-dim_ball_volume_and_surface}
%     V_d(r) = \frac{\pi^{d/2}}{\Gamma(d/2 + 1)} \cdot r^d \qquad S_d(r) = 2 \pi V_{d-1}(r)
% \end{equation}
% Отсюда имеем
% \[
%     r = \frac{1}{\sqrt{\pi}} \left[ \Gamma(d/2+1) V_d(r) \right]^{1/d},
% \]
Найдём моменты случайной величины $ r_k $.
Вспомним, что объём $ d $-мерного шара радиуса $ r $ пропорционален $ r^d $:
\[
    V_d(r) = V_d(1) \cdot r^d \quad \Longrightarrow \quad r = \sqrt[d]{V_d(r) / V_d(1)},
\]
Тогда
%
{\allowdisplaybreaks
\begin{align}
    \label{eq:low_concentration:kNN_distance_expectation}
    \expect r_k^m &= \int\limits_0^{+\infty} x^m \rho_{r_k}(x) \, dx =
    \int\limits_0^{+\infty} x^m e^{-V(x) n} \frac{(V(x) n)^{k-1}}{(k-1)!} \, S(x) n \, dx = \\
    &= \int\limits_0^{+\infty} x^m(V) \cdot e^{-V n} \frac{(V n)^{k-1}}{(k-1)!} \, d(V n) = \\
    &= V(1)^{-m/d} \int\limits_0^{+\infty} V^{m/d} \cdot e^{-V n} \frac{(V n)^{k-1}}{(k-1)!} \, d(V n) = \\
    &= \frac{(V(1) n)^{-m/d}}{(k-1)!} \int\limits_0^{+\infty} z^{m/d + k - 1} e^{-z} dz
    = \frac{\Gamma(m/d + k)}{\Gamma(k)} (V(1) n)^{-m/d}
\end{align}
}
В частности, в трёхмерном случае среднее расстояние до ближайшего соседа примерно равно $ 0.83 \cdot n^{-1/3} $.

% \begin{figure}
%     \centering
%     \begin{subfigure}[t]{0.5\textwidth}
%         \centering
%         \begin{gnuplot}[terminal=tikz, terminaloptions={color size 8.5cm,5.0cm}]
%             load "./gnuplot/common.gp"
%             load "./gnuplot/kNN_distribution.gp"
% 
%             set yrange [0:1]
% 
%             f_1(x)  = F_r_normalized(x,3,1)
%             f_4(x)  = F_r_normalized(x,3,4)
%             f_10(x) = F_r_normalized(x,3,10)
% 
%             plot [-3:3] f_1(x) lw 2 t "$ F_{r_1}(x) $", \
%                         f_4(x) lw 2 t "$ F_{r_4}(x) $", \
%                         f_10(x) lw 2 t "$ F_{r_{10}}(x) $"
%         \end{gnuplot}
%         \caption{Характерный вид функций распределения $ F_{r_k}(x) $.}
%     \end{subfigure}%
%     \begin{subfigure}[t]{0.5\textwidth}
%         \centering
%         \begin{gnuplot}[terminal=tikz, terminaloptions={color size 8.5cm,5.0cm}]
%             load "./gnuplot/common.gp"
%             load "./gnuplot/kNN_distribution.gp"
% 
%             f_1(x)  = rho_r_normalized(x,3,1)
%             f_4(x)  = rho_r_normalized(x,3,4)
%             f_10(x) = rho_r_normalized(x,3,10)
% 
%             plot [-3:3] f_1(x) lw 2 t "$ \\rho_{r_1}(x) $", \
%                         f_4(x) lw 2 t "$ \\rho_{r_4}(x) $", \
%                         f_10(x) lw 2 t "$ \\rho_{r_{10}}(x) $"
%         \end{gnuplot}
%         \caption{Характерный вид плотности распределения $ \rho_{r_k}(x) $.}
%     \end{subfigure}
% \end{figure}


\section{Модель тромбоцитарного тромба}
\label{section:blood:white_clot_model}

В данном разделе будет предложена модель образования белого тромба,
дополняющие результаты работ~\cite{bouchnita2020mathematical, vassilevski2020parallel}.
Основу модели будут составлять три компоненты: свободные, липкие и закрепившиеся тромбоциты.
Первые два компартмента будут моделироваться пассивной примесью,
третий будет образовывать неподвижную пористую среду.
Реакционная часть модели позволит описать переход тромбоцитов из одного компартмента в другой
под действием слипания и разъединения потоком.
Модели вязкости и проницаемости слипшейся массы должны позволить воспроизвести наблюдаемые процессы закрепления тромба
и блокировки потока крови.

Программная имплементация основана на конечно-объемном коде
уравнения переноса-диффузии-реакции из работы~\cite{vassilevski2020parallel},
позволяющем модифицировать реакционную часть,
отключать перенос определённых примесей,
а также динамически менять вязкость и проницаемость среды.
Поэтому в настоящем разделе мы сфокусируемся только на существенных модификациях,
вносимых в используемую модель.
Конкретно, будут описаны новые реакционные члены, а также модель вязкости и проницаемости.

\subsection{Реакционная часть}
\label{subsection:white_clot_model:reactions}

Традиционно процесс образования белого тромба описывается
многокомпонентной моделью~\cite{sorensen1999platelets_deposition_model, goodman2005thrombosis_model, taylor2016thrombosis_model, wu2017deposition_model}.
Однако экспериментальные данные позволяют предположить,
что воспроизведение отдельно механизма слипания тромбоцитов может потребовать добавления
лишь небольшого числа новых компонент в существующую модель фибринового тромба.

Результаты лабораторных опытов говорят о том,
что в областях повышенных сдвиговых напряжений происходят качественные изменения тромбоцитов,
заставляющие их налипать на стенку ниже по течению,
где сдвиговое напряжение возвращается к нормальному значению~\cite{rahman2019platelet_adhesion}.
Также известно, что сдвиговое напряжение влияет и на местную агрегацию тромбоцитов:
при его повышении фибриновые и фибриногеновые связи наблюдаются реже,
а связи посредством фактора фон Виллебранда~--- чаще~\cite{savage1996platelet_adhesion}.
Это связано с механическими свойствами фактора фон Виллебранда:
при малых скоростях сдвига фактор свёрнут в глобулу
и не может сцеплять тромбоциты между собой и со стенкой,
однако высокое сдвиговое напряжение разворачивает глобулы и открывает сайты связывания,
что в конечном итоге приводит к агрегации тромбоцитов~\cite{lippok2016vWF_unfolding, schneider2007vWF_unfolding, zhussupbekov2021vWF_unfolding, avtaeva2022vWF}.

Можно выдвинуть предположение,
что в областях с повышенным сдвиговым напряжением тромбоциты образуют липкие агрегаты,
скреплённые частично или полностью развёрнутым фактором фон Виллебранда.
Агрегаты могут далее оседать как в месте своего образования, так и ниже по течению.
Такая модель требует введения всего двух компонент:
концентрации липких и слипшихся тромбоцитов.
Требуется также учесть разрушение связей между тромбоцитами из-за механического воздействия
или других факторов.
Итоговые уравнения имеют вид
%
\begin{equation}
    \label{eq:reactions:general_reactions_equations}
    \begin{dcases}
        \frac{\partial \phi_f}{\partial t} &= -k_{f \to a} \cdot \phi_f + k_{\mathrm{da}} \cdot \phi_a \\
        \frac{\partial \phi_a}{\partial t} &=  k_{f \to a} \cdot \phi_f - k_{a \to d} \cdot \phi_a + k_{\mathrm{da}} \cdot (\phi_d - \phi_a) \\
        \frac{\partial \phi_d}{\partial t} &=  k_{a \to d} \cdot \phi_a - k_{\mathrm{da}} \cdot \phi_d \\
    \end{dcases}
\end{equation}
%
где $ \phi_f $, $ \phi_a $ и $ \phi_d $~--- концентрации свободных, липких и слипшихся тромбоцитов соответственно,
$ k_\mathrm{da} $~--- скорость дизагрегации тромбоцитов из-за внешнего воздействия,
а $ k_{f \to a} $ и $ k_{a \to d} $ отвечают за скорость агрегации и слипания тромбоцитов.
Концентрации $ \phi_f $ и $ \phi_a $ участвуют в переносе и диффузии в качестве пассивной примеси,
$ \phi_d $ же считается концентрацией строго неподвижной примеси.

Будем рассматривать случай высоких сдвиговых напряжений,
когда большая часть связей между тромбоцитами опосредована фактором фон Виллебранда.
Предполагая объёмную концентрацию $ \phi_a $ постоянной (равновесной),
можно аналитически решить~\eqref{eq:reactions:general_reactions_equations} для зависимости $ \phi_d $ от времени:
%
\begin{equation}
    \label{eq:reactions:phi_d_closed_form}
    \frac{\partial \phi_d}{\partial t} = k_{a \to d} \cdot \phi_a - k_{\mathrm{da}} \cdot \phi_d
    \quad \Longrightarrow \quad \phi_d(t) = \phi_a \frac{k_{a \to d}}{k_\mathrm{da}} - e^{-k_{\mathrm{da}}} \left( \phi_a \frac{k_{a \to d}}{k_\mathrm{da}} - \phi_d(0) \right)
\end{equation}
%
В работе~\cite{savage1996platelet_adhesion} представлены данные,
описывающие динамику тромбоцитов, налипших на подложку с нанесённым фактором фон Виллебранда,
с течением времени и при разных сдвиговых напряжениях.
В результате решения задачи наименьших квадратов для полученной экспоненциальной модели~\eqref{eq:reactions:phi_d_closed_form}
можно сделать вывод, что $ k_\mathrm{da} $ почти не зависит от сдвигового напряжения
(что не так для фибрина и фибриногена) и примерно равен $ 3.3 \cdot 10^{-2} \; \text{с}^{-1} $~---
см. рисунок~\ref{fig:reactions:vWF_platelets_count_time}.

\begin{figure}[ht!]
    \centering
    \small
    \begin{gnuplot}[terminal=tikz, terminaloptions={color size 16.0cm,10.0cm fontscale 0.8}]
        load './gnuplot/common.gp'

        set datafile separator ' '
        set key vertical maxrows 6

        set style increment default
        set style data lines
        set xlabel  "$ t, \\; \\textnormal{мин} $"
        set xrange  [ 0 : 5 ] noreverse writeback
        set ylabel  "$ N_\\textnormal{platelets}, \\; \\textnormal{шт.} $" offset -1 #rotate by 0
        set yrange  [ 0 : 1600 ] noreverse writeback

        # Модель
        f(a, k, x) = a * (1.0 - exp(-k * x))

        set xtics 1
        #set xzeroaxis lw 2

        plot "./data/platelets/Savage_fibrinogen_50_time_data.csv"   using 1:2 with points ps 3 t "Фибриноген, $ 50 \\; \\textnormal{с}^{-1} $", \
             "./data/platelets/Savage_fibrinogen_630_time_data.csv"  using 1:2 with points ps 3 t "Фибриноген, $ 630 \\; \\textnormal{с}^{-1} $", \
             "./data/platelets/Savage_fibrinogen_1500_time_data.csv" using 1:2 with points ps 3 t "Фибриноген, $ 1500 \\; \\textnormal{с}^{-1} $", \
             "./data/platelets/Savage_vWF_50_time_data.csv"          using 1:2 with points ps 3 t "vWF, $ 50 \\; \\textnormal{с}^{-1} $", \
             "./data/platelets/Savage_vWF_630_time_data.csv"         using 1:2 with points ps 3 t "vWF, $ 630 \\; \\textnormal{с}^{-1} $", \
             "./data/platelets/Savage_vWF_1500_time_data.csv"        using 1:2 with points ps 3 t "vWF, $ 1500 \\; \\textnormal{с}^{-1} $", \
             f(556.11,  0.4353, x) with lines lc 1 lw 2 t "$ k_\\mathrm{da} = 7.25 \\cdot 10^{-3} \\; \\textnormal{с}^{-1} $", \
             f(2013.02, 0.0296, x) with lines lc 2 lw 2 t "$ k_\\mathrm{da} = 4.92 \\cdot 10^{-4} \\; \\textnormal{с}^{-1} $", \
             f(44.27,   2.2744, x) with lines lc 3 lw 2 t "$ k_\\mathrm{da} = 3.71 \\cdot 10^{-2} \\; \\textnormal{с}^{-1} $", \
             f(460.91,  2.1932, x) with lines lc 4 lw 2 t "$ k_\\mathrm{da} = 3.65 \\cdot 10^{-2} \\; \\textnormal{с}^{-1} $", \
             f(799.96,  1.6614, x) with lines lc 5 lw 2 t "$ k_\\mathrm{da} = 2.76 \\cdot 10^{-2} \\; \\textnormal{с}^{-1} $", \
             f(980.54,  2.4312, x) with lines lc 6 lw 2 t "$ k_\\mathrm{da} = 4.05 \\cdot 10^{-2} \\; \\textnormal{с}^{-1} $"
    \end{gnuplot}
    \caption{Зависимость числа налипших тромбоцитов от времени при разной сдвиговой скорости в потоке у стенки~\cite{savage1996platelet_adhesion}.}
    \label{fig:reactions:vWF_platelets_count_time}
\end{figure}

Из той же работы также можно получить зависимость $ k_{f \to a} $ и $ k_{a \to d} $ от сдвигового напряжения,
которая должна быть пропорциональна предельному числу налипших на подложку тромбоцитов
(см.~\eqref{eq:reactions:phi_d_closed_form}).
%(рисунок 2 в~\cite{savage1996platelet_adhesion} и рисунок~\ref{fig:reactions:vWF_platelets_count}).
Решая задачу наименьших квадратов с экспоненциальной моделью, получаем
%
\begin{equation}
    \label{eq:reactions:vWF_shear_model}
    k_{f \to a}, \; k_{a \to d} \sim 1 - a \, e^{-b \, \dot{\gamma}},
    \qquad
    a = 0.76, \;
    b = 2.85 \cdot 10^{-3} \; \textnormal{с},
\end{equation}
%
где $ \dot{\gamma} $~--- скорость сдвига.
Экспериментальные данные и кривая модели приведены на рисунке~\ref{fig:reactions:vWF_platelets_count}.

\begin{figure}[ht!]
    \centering
    \small
    \begin{gnuplot}[terminal=tikz, terminaloptions={color size 16.0cm,8.0cm fontscale 0.8}]
        load './gnuplot/common.gp'

        set datafile separator ' '

        set style increment default
        set style data lines
        set xlabel  "$ \\dot \\gamma, \\; \\textnormal{с}^{-1} $"
        set xrange  [ 0 : 1600 ] noreverse writeback
        set ylabel  "$ N_\\textnormal{platelets}, \\; \\textnormal{шт.} $" offset -1 #rotate by 0
        set yrange  [ 0 : 1000 ] noreverse writeback

        # Модель
        f(x) = 805.7 * (1.0 - 0.759 * exp(-2.845e-3 * x))

        set xtics 200
        #set xzeroaxis lw 2

        plot "./data/platelets/Savage_data.csv" using 1:3 with points ps 3 t "экспериментальные данные \\cite{savage1996platelet_adhesion}", \
             f(x) with lines lw 3 t "модель \\eqref{eq:reactions:vWF_shear_model}"
    \end{gnuplot}
    \caption{Зависимость числа налипших тромбоцитов от сдвиговой скорости в потоке у стенки~\cite{savage1996platelet_adhesion}.}
    \label{fig:reactions:vWF_platelets_count}
\end{figure}

В работе~\cite{wu2017deposition_model} предлагается считать процесс формирования тромба послойным,
то есть переход $ \phi_a \to \phi_d $ возможен только вблизи поверхности сосуда или уже образовавшегося тромба.
Математически данная идея формализуется посредством свёртки некоторого ядра $ K(\vec{r}) $,
ширина которого отвечает за характерное расстояние слипания тромбоцитов,
с функцией $ \alpha_c \cdot \phi_d(\vec{r}) + \alpha_b \cdot \indicator_{\partial \Omega_{\mathrm{adh}}}(\vec{r}) $,
где $ \partial \Omega_\mathrm{adh} $ обозначает границу области, на которую могут крепиться тромбоциты,
а коэффициенты $ \alpha $ регулируют скорость налипания.

На практике расстояния, на которых формируются связи между тромбоцитами,
гораздо меньше характерных размеров ячеек расчётной сетки,
а потому результат свёртки будет равен $ \alpha_c \cdot S_d + \alpha_b \cdot S_b $,
где $ S_b $~--- суммарная площадь контакта ячейки с той частью границы, на которую могут крепиться тромбоциты,
а $ S_c $~--- суммарная площадь контакта с соседними ячейками, в которых тромб уже сформирован.
В работе~\cite{wu2017deposition_model} предлагается следующая формула для $ S_c $:
\[
    S_c(\mathrm{c}) = \sum_{\mathrm{f} \in \textnormal{faces}(\mathrm{c})} |\mathrm{f}| \cdot
    (\phi_d(\mathrm{f}) / \phi_{\max}) \cdot \indicator_{[\phi_\textnormal{clot}; +\infty)}(\phi_d(\mathrm{f})),
\]
где $ \mathrm{c} $~--- рассматриваемая ячейка,
%$ |\mathrm{c}| $~--- объём ячейки $ \mathrm{c} $,
$ \textnormal{faces}(\mathrm{c}) $~--- грани ячейки $ \mathrm{c} $,
$ |\mathrm{f}| $~--- площадь грани $ \mathrm{f} $,
$ \phi_d(\mathrm{f}) $~--- концентрация $ \phi_d $
в ячейке через грань $ \mathrm{f} $ от $ \mathrm{c} $,
$ \phi_{\max} $~--- максимально допустимая концентрация тромбоцитов
(см.~\ref{subsection:volume_distribution_models:high_concentration}),
а $ \phi_{\textnormal{clot}} $~--- минимальная концентрация тромбоцитов в белом тромбе
(определяется экспериментально).


\subsection{Модель вязкости}
\label{subsection:white_clot_model:viscocity}

В работе~\cite{jamiolkowski2016visualization} было продемонстрировано,
что белый тромб ведёт себя как вязкопластичная среда,
вязкость которой растёт по мере роста концентрации за счёт образования новых связей между тромбоцитами.
Пренебрегая пластичным аспектом,
построим модель вязкости тромба из предположения о её пропорциональности числу связей между липкими ($ \phi_a $) тромбоцитами;
связи между уже слипшимися и закрепившимися тромбоцитами не рассматриваются, так как последние формируют неподвижную массу,
не участвующую в переносе и диффузии.

Микрофотографии, приведённые в работе~\cite{schneider2007vWF_unfolding},
говорят о том, что агрегация тромбоцитов посредством
фактора фон Виллебранда может наблюдаться на расстоянии до $ 10 - 15 \; \text{мкм} $;
обозначим данное характерное расстояние за $ L_\mathrm{vWF} $ .
При этом можно предполагать, что сравнительно устойчивые агрегаты формируются только тогда,
когда каждый тромбоцит сцеплен с тремя или более другими тромбоцитами.
Тогда можно рассмотреть следующую модель вязкости:
%
\begin{equation}
    \label{eq:viscocity:viscocity_model}
    \nu = \nu_0 \cdot (1 + \beta \cdot F_{r_k}(L_\mathrm{vWF})),
\end{equation}
%
где $ \nu_0 $~--- вязкость крови без агрегаций из тромбоцитов,
$ F_{r_k} $ определена в уравнении~\eqref{eq:low_concentration:kNN_distance_CDF},
$ \beta $~--- параметр влияния степени агрегации тромбоцитов на вязкость,
$ k \approx 4 $, $ L_\mathrm{vWF} \approx 10 \; \textnormal{мкм} $.
В действительности $ \beta $ также зависит от скорости сдвига
(нелинейная вязкость), однако данных для восстановления зависимости не хватает;
в качестве начального приближения можно взять $ \beta \approx 5 - 20 $~\cite{ranucci2014clot_viscocity}.


\subsection{Модель проницаемости}
\label{subsection:white_clot_model:permeability}

Вопрос о проницаемости и пористой структуре тромба подробно исследован в работе~\cite{wufsus2013clot_permeability}.
Было показано, что модель Кожены-Кармана~\cite{warrenl1976chemical_engineering}
достаточно хорошо описывает зависимость проницаемости тромба от концентрации тромбоцитов:
%
\begin{equation}
    \label{eq:permeability:Kozeny_Carmen}
    k_p = \frac{\Psi_p^2 \cdot d_p^2 \cdot (1 - \phi_d)^3}{150 \cdot \phi_d^3},
\end{equation}
%
где $ \Psi_p $~--- сферичность тромбоцитов ($ \approx 0.7 $~\cite{wufsus2013clot_permeability}),
$ d_p $~--- диаметр тромбоцитов.
