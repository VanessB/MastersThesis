\begin{abstract}
    \begin{center}
        \large{Решение жестких систем реакций свертывания крови и моделирование образования тромба в придатке левого желудочка} \\
    \large\textit{Бутаков Иван Дмитриевич} \\[1 cm]
    \end{center}

    При патологиях сердечно-сосудистой системы характер течения крови меняется,
    повышается риск тромбообразования.
    Для моделирования процессов коагуляции требуется решать систему переноса"=диффузии"=реакции,
    где реакционная часть представлена жёсткой системой каскада свёртывания крови.
    Применение традиционных численных схем при интегрировании данной системы может вести к неустойчивости,
    соответствующие численные решения могут оказаться нефизичными.
    Также большинство существующих моделей каскада свёртывания крови концентрируются на отдельных сторонах процесса,
    что существенно ограничивает их область применения.
    В данной работе предложен новый подход для построения неявных методов численного интегрирования жёстких нелинейных систем.
    Полученные методы лучше адаптированы к определённому классу задач
    и позволяют динамически поддерживать баланс между устойчивостью численного решения
    и линейностью возникающих на каждом шаге алгебраических уравнений.
    Полученные методы были проверены на следующих жёстких системах:
    модель Лотки-Вольтерры, модель осциллятора Ван дер Поля, модель каскада свёртывания крови.
    Во всех случаях предложенные методы показали улучшение устойчивости в сравнении со стандартными подходами к аппроксимации:
    неявным методом Эйлера и методом трапеций.
    Наконец, в работе предложена модификация существующей модели фибринового тромба,
    позволяющая также воспроизводить рост тромбоцитарного тромба
    в областях с особыми характеристиками потока крови.

    \vfill

    \textbf{Abstract}% \\[1 cm]

    Solving stiff blood coagulation system and modeling white clot formation
\end{abstract}
\newpage
