%% Равенство по определению
\def\defeq{\triangleq}

%% Функции
\def\blankarg{\,\cdot\,}

%% Множества.
\def\naturals{\mathbb{N}}  % Натуральные числа.
\def\integers{\mathbb{Z}}  % Целые числа.
\def\reals{\mathbb{R}}     % Действительные числа
\def\complexes{\mathbb{C}} % Комплексные числа.

%% Линейная алгебра.
\def\diag{\operatorname{diag}} % Диагональная матрица.

%% Математический анализ.
\newcommand{\limto}[2]{\underset{#1 \to #2}{\longrightarrow}}  % Стрелка, обозначающая предел.
\newcommand{\overlimto}[3]{\overset{#3}{\limto{#1}{#2}}}       % Стрелка, обозначающая предел, с подписью типа сходимости.
\def\banachspace{\mathcal{B}}                                  % Банахово пространство.
\def\hilbertspace{\mathcal{H}}                                 % Гильбертово пространство.
\def\linopset{\mathcal{L}}                                % Множество линейных ограниченных операторов.
\newcommand{\dotprod}[2]{\left\langle #1, #2 \right\rangle}    % Скалярное произведение.

\DeclareMathOperator*{\argmin}{\arg\min}

%% Комплексные числа.
\def\Re{\operatorname{Re}} % Действительная часть.
\def\Im{\operatorname{Im}} % Мнимая часть.

%% Асимптотические классы.
\def\Oclass{\mathcal{O}} % О-большое.

%% Выделение определения.
\DeclareTextFontCommand{\defemph}{\bfseries\em}
