\uchapter{Сокращения, обозначения, термины и определения}
\label{chapter:definitions} \index{Обозначения}

\begin{center}
    \begin{tabularx}{\textwidth}{cl}
        $ \defarr $                      & <<\ldots по определению тогда и только тогда, когда \ldots>> \\
        $ \defeq $                       & <<\ldots по определению равно \ldots>> \\
        \rule{0pt}{16pt}%
        $ \banachspace $                 & Банахово пространство. \\
        $ \hilbertspace $                & Гильбертово пространство. \\
        %$ x^* $                          & эрмитово (в случае числа~--- комплексное) сопряжение. \\
        \rule{0pt}{16pt}%
        $ R(z) $                         & функция устойчивости численного метода. \\
        $ \stabreg $                     & область устойчивости численного метода. \\
        $ F $                            & \makecell[l]{матрица якоби правой части автономной системы \\
                                                        обыкновенных дифференциальных уравнений:
                                                        $ F(\bvec{x}) \defeq \frac{\partial f}{\partial \bvec{x}}(\bvec{x}) $.} \\
        $ \res $                         & невязка численного метода. \\
        $ \jac $                         & матрица Якоби невзязки численного метода:
                                           $ \jac(\bvec{x}) \defeq \frac{\partial \res(\bvec{x})}{\partial \bvec{x}} $. \\
        \rule{0pt}{16pt}%
        $ (\Omega, \setfamily, \proba) $ & \makecell[l]{вероятностное пространство ($ \Omega $~--- множество исходов, \\
                                                        $ \setfamily $~--- $ \sigma $-алгебра, $ \proba $~--- вероятностная мера).} \\
        %$ \borel(A) $, $ \borel_A $      & \makecell[l]{Борелевская $ \sigma $-алгебра, определённая на множестве $ A $ (если $ A $ не указано, \\ по умолчанию предполагается $ A = \reals $).} \\
        $ \indicator_A $                 & индикаторная функция множества $ A $. \\
        $ \expect X $                    & математическое ожидание случайной величины $ X $. \\
        $ \dispersion X $                & дисперсия случайной величины $ X $. \\
        $ \rvcenter X $                  & <<центрированная>> случайная величина: $ \rvcenter X = X - \expect X $. \\
    \end{tabularx}
\end{center}

Нижний индекс может быть использован для обозначения принадлежности.
Например, $ R_M(z) $~--- функция устойчивости численного метода~$ M $.
