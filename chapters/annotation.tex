\begin{abstract}
    \begin{center}
        \large{Решение жестких систем реакций свертывания крови и моделирование образования тромба в придатке левого желудочка} \\
    \large\textit{Бутаков Иван Дмитриевич} \\[1 cm]
    \end{center}

    При патологиях в сердце характер течения в придатке предсердия левого желудочка меняется,
    повышается риск образования в нем тромба.
    Для моделирования процессов тромбообразования требуется решать систему переноса-диффузии-реакции,
    где реакционная часть представлена жёсткой системой каскада свёртывания крови.
    Применение традиционных численных схем при интегрировании данной системы может вести к неустойчивости,
    соответствующие численные решения могут оказаться нефизичными.
    %нефизичным осцилляциям, к отсутствию сходимости итерационных методов решения возникающих нелинейных уравнений.
    В данной работе предложены два метода для неявного численного интегрирования жёстких нелинейных систем,
    способные решить указанные проблемы:
    модифицированный метод Ньютона и взвешенный метод Эйлера.
    Методы основаны на <<фильтрации>> спектра матрицы Якоби правой части системы.
    Фильтрация производится путём комбинации явного и неявного метода Эйлера с матричным весовым коэффициентом
    и позволяет разделить составляющие спектра, имеющие разный знак действительной части.
    %Весовая матрица вычисляется путём применения специально подобранной функции к спектру матрицы Якоби правой части.
    %Подбор функции осуществлён с целью получения экспоненциального интегратора.
    Подбор весовой матрицы осуществлён с целью получения экспоненциального интегратора.
    Полученные методы были проверены на следующих жёстких системах:
    модель Лотки-Вольтерры, модель осциллятора Ван дер Поля, модель каскада свёртывания крови.
    Во всех случаях предложенные методы показали улучшение устойчивости в сравнении со стандартными подходами к аппроксимации:
    неявным методом Эйлера и методом трапеций.
    В дополнение к этому модифицированный метод Ньютона позволил кратно увеличить шаг интегрирования системы по времени, что критически важно для практических задач.
    %В работе также описаны детали программной реализации предложенных методов.
    %Наконец, дан краткий обзор возможностей по применению предложенной техники для построения других методов интегрирования, обладающих схожими свойствами.
    %Наконец, описано дальнейшее применение предложенных методов к практической задаче.

    \vfill

    \textbf{Abstract}% \\[1 cm]

    Solving stiff blood coagulation system and modeling clot formation in left atrial appendage
\end{abstract}
\newpage
