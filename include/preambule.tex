%% Пакеты для работы с математикой.
\usepackage{amsmath,amsfonts,amssymb,amsthm,mathtools}
\usepackage{icomma}
\usepackage{nicematrix} % Красивые матрицы.

%% Работа с русским языком.
\usepackage{cmap}            % поиск в PDF
%\usepackage{mathtext}       % русские буквы в формулах
\usepackage[T2A]{fontenc}    % кодировка
\usepackage[utf8]{inputenc}  % кодировка исходного текста
\usepackage[russian]{babel}  % локализация и переносы

%% Номера формул.
\mathtoolsset{showonlyrefs=true} % Показывать номера только у тех формул, на которые есть \eqref{} в тексте.
%\usepackage{leqno}               % Немуреация формул слева

%% Шрифты.
%\usepackage{euscript}	 % Шрифт Евклид
%\usepackage{mathrsfs}   % Красивый матшрифт

%\usepackage{fontspec}
%\defaultfontfeatures{Ligatures={TeX},Renderer=Basic} 
%\setmainfont[Ligatures={TeX,Historic}]{Times New Roman}

%\usepackage{mathptmx}
%\usepackage{newtxtext,newtxmath}

%% Работа с картинками.
\usepackage[labelsep=endash]{caption} % Пакет для подписей к рисункам, в частности, для работы caption*.
\usepackage{subcaption}               % Подкартинки.
\usepackage{graphicx}                 % Для вставки рисунков.
\usepackage{wrapfig}                  % Обтекание рисунков и таблиц текстом.

\graphicspath{{images/}}       % Папки с картинками.

%% Работа с таблицами.
%\usepackage{array,tabularx,tabulary,booktabs} % Дополнительная работа с таблицами.
%\usepackage{longtable}                        % Длинные таблицы.
%\usepackage{multirow}                         % Слияние строк в таблице.

% Ячейки на несколько строк.
\usepackage{multirow}
\usepackage{makecell}
\usepackage{tabularx} % Таблица с регулируемой шириной столбцов и работающими сносками.

%% TikZ.
\usepackage{tikz}
%\usetikzlibrary{graphs,graphs.standard}

%% Графики gnuplot.
\usepackage[shell, subfolder, cleanup]{gnuplottex}
\usepackage{gnuplot-lua-tikz}

%% Перенос знаков в формулах (по Львовскому).
\newcommand*{\hm}[1]{#1\nobreak\discretionary{}{\hbox{$\mathsurround=0pt #1$}}{}}

%% Дополнения.
\usepackage{float}             % Добавляет возможность работы с командой [H] которая улучшает расположение на странице.
\usepackage{textcomp, gensymb} % Красивые градусы.

%% Вращение pdf или его наполнения.
\usepackage{rotating}  % Вращение плавающих объектов.
\usepackage{pdflscape} % Вращение страницы.

%% Объекты на одтельной странице.
%\usepackage{afterpage}

%% Контроль положения плавающих объектов.
\usepackage{placeins}

%% hyperref для ссылок внутри  pdf.
\usepackage[unicode, pdftex]{hyperref}

%% Список литературы.
\usepackage[doi=false, isbn=false, url=false, eprint=false, backend=biber, style=gost-numeric]{biblatex}
\usepackage{csquotes}

%% Удобная работа с названием работы.
\usepackage{titling}

%% Удобная работа с интервалами.
\usepackage{setspace}

%% Подчёркивание с переносом.
\usepackage{ulem}
\normalem
