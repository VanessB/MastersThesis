%% Поля (геометрия страницы).
\usepackage[left=3cm, right=1.5cm, top=2cm, bottom=2cm, bindingoffset=0cm]{geometry}
%headheight=28pt

%% Работа с картинками.
\captionsetup{justification=centering} % Центрирование подписей к картинкам.
\captionsetup[figure]{name=Рисунок}    % Полное наименование рисунка.
\setlength\fboxsep{3pt}                % Отступ рамки \fbox{} от рисунка.
\setlength\fboxrule{1pt}               % Толщина линий рамки \fbox{}.

%% Русские списки.
\usepackage[shortlabels]{enumitem}
\makeatletter
\AddEnumerateCounter{\asbuk}{\russian@alph}{щ}
\makeatother

%% Интервалы.
\linespread{1.5}              % Междустрочный интервал.
\setlength\parindent{1.25cm}  % Абзацный отступ.
\usepackage{indentfirst}      % Отступ в первом абзаце.
%\setlist[enumerate,1]{leftmargin=1.25cm} % Отступ в списках.
%\setlist[itemize,1]{leftmargin=1.25cm}

%% Верхний колонтитул.
\usepackage{fancyhdr}
%\pagestyle{fancy}
%\setlength{\headheight}{23pt}

%% Стиль доказательства.
\let\oldproofname=\proofname \renewcommand{\proofname}{\rm\bf{\oldproofname}}


%% Стили заголовков.
\usepackage{titlesec}
% Глава.
%\titleformat{\chapter}{\centering\normalfont\fontsize{14}{15}\bfseries}{\chaptertitlename\ \thechapter:}{1em}{}
\titleformat{\chapter}{\centering\normalfont\fontsize{14}{15}\bfseries\MakeUppercase}{\thechapter}{1em}{}
\titlespacing*{\chapter}{0pt}{0pt}{10pt}
% Секция.
\titleformat{\section}{\centering\normalfont\fontsize{14}{15}\bfseries}{\thesection}{1em}{}
% Подсекция.
\titleformat{\subsection}{\centering\normalfont\fontsize{14}{15}\bfseries\itshape}{\thesubsection}{1em}{}

% Глава без номера.
\newcommand{\uchapter}[1]{\chapter*{#1}
\addcontentsline{toc}{chapter}{\protect\numberline{}#1}}
