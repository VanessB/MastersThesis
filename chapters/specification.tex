\begin{spacing}{1.0}
\fontsize{12}{14.5}\selectfont

\begin{center}
    %% *название института*
    Автономная некоммерческая образовательная организация \\ высшего образования \\
    \textbf{<<НАУЧНО-ТЕХНОЛОГИЧЕСКИЙ УНИВЕРСИТЕТ <<СИРИУС>>} \\
    \vspace{0.66\baselineskip}
    Научный центр информационных технологий и искусственного интеллекта \\
    \vspace{0.66\baselineskip}
    направление <<Математическое моделирование в биомедицине и геофизике>> \\

    \vspace{1.5\baselineskip}

    \begin{tabularx}{\textwidth}{>{\hsize=.48\hsize}Y>{\hsize=.04\hsize}X>{\hsize=.48\hsize}Y}
        & & \signatureblock{
            \MakeUppercase{К защите допустить} \vspace{0.5\baselineskip} \newline
                Научный руководитель направления <<Математическое моделирование в биомедицине и геофизике>> \newline
                д-р. физ.-мат. наук, профессор, член корреспондент РАН
            }
            {Ю.\,В.\,Василевский}
    \end{tabularx}

    \vspace{2.0\baselineskip}

    \textbf{\MakeUppercase{Техническое задание}} \\
    \textbf{на выполнение выпускной квалификационной работы}

    по направлению подготовки 01.04.02 Прикладная математика и информатика \\
    (направленность (профиль) <<Математическое моделирование в биомедицине и нефтегазовом инжиниринге>>) \\

    \vspace{\baselineskip}

    Бутаков Иван Дмитриевич

\end{center}

\begin{enumerate}[itemsep=0.0\baselineskip]
    \item Тема: <<\thetitle>>
    \item Цель: Разработка математических моделей и программного обеспечения для моделирования каскада реакций свёртывания крови.
    \item Задачи: Разработка и валидация математических моделей и программного обеспечения
        для моделирования каскада реакций свёртывания крови,
        анализ возможности интеграции полученных результатов в существующие модели тромбообразования.
    \item Рабочий график (план) выполнения выпускной квалификационной работы:

        \begin{tabularx}{\linewidth}{|c|>{\hsize=.6\hsize}X|>{\hsize=.3\hsize}X|}
            \hline
            \textbf{\textnumero} & \textbf{Перечень заданий} & \textbf{Сроки выполнения} \\
            \hline
            1 & Анализ существующих моделей тромбообразования & 30.01.2024 -- 15.02.2024 \\
            \hline
            2 & Анализ литературы, посвященной численному интегрированию жёстких систем реакций & 15.02.2024 -- 01.03.2024 \\
            \hline
            3 & Разработка математической модели свёртывания крови, составление набора валидационных экспериментов & 01.03.2024 -- 01.04.2024 \\
            \hline
            4 & Программная имплементация разработанной математической модели & 01.04.2024 -- 01.05.2024 \\
            \hline
            5 & Валидация полученного программного комплекса на собранном наборе проверочных экспериментов & 01.05.2024 -- 23.05.2024 \\
            \hline
        \end{tabularx}
\end{enumerate}

%\noindent
%Дата выдачи: \dateblock

\begin{center}
    \begin{tabularx}{\textwidth}{>{\hsize=.6\hsize}Y>{\hsize=.4\hsize}Y}
        Дата выдачи: \dateblock & \\
    \end{tabularx}

    \begin{tabularx}{\textwidth}{>{\hsize=.48\hsize}Y>{\hsize=.04\hsize}X>{\hsize=.48\hsize}Y}
        Руководитель: & & \\
        \signatureblock{
                к.ф.-м.н., снс ИВМ РАН
        }
        {К.\,М.\,Терехов} & & \\
    \end{tabularx}

    \vspace{\baselineskip}

    \begin{tabularx}{\textwidth}{>{\hsize=.48\hsize}Y>{\hsize=.04\hsize}X>{\hsize=.48\hsize}Y}
        Задание принял к исполнению: & & \\
        \signatureblock{Студент гр. М01ММ-22}
        {И.\,Д.\,Бутаков} & &
    \end{tabularx}

    \vspace{\fill}
\end{center}

\normalsize
\end{spacing}
